Putting all things together, result on the SM Higgs search will be discussed in this chapter. 

\section{Cut-based analysis results}  
%%%%% cut-based result %%%%%

Fig.~\ref{fig:cutbased125_0jet} and \ref{fig:cutbased125_0jet} show 
\mT\ and \mll\ distributions for \mHi=125~\GeV\ analysis in 0-jet and 1-jet categories, 
respectively. 
All lepton final states are shown, \DF, \SF and inclusive channel starting from the top. 
Tab.~\ref{tab:cut7tev} and \ref{tab:cut8tev} show the yields of each process in 7 and 8~\TeV, 
repectively.
The data shows a good agreement with the assumption of \mHi=125~\GeV\ Higgs boson. 

\begin{figure}[htp] 
\centering 
\begin{tabular}{c} 
\includegraphics[width=0.45\textwidth]{figures/hww_analysis17_125_ALL_of_0j_mt.pdf}
\includegraphics[width=0.45\textwidth]{figures/hww_analysis17_125_ALL_of_0j_mll.pdf} 
\\
\includegraphics[width=0.45\textwidth]{figures/hww_analysis17_125_ALL_sf_0j_mt.pdf}
\includegraphics[width=0.45\textwidth]{figures/hww_analysis17_125_ALL_sf_0j_mll.pdf}
\\
\includegraphics[width=0.45\textwidth]{figures/hww_analysis17_125_ALL_incl_0j_mt.pdf}
\includegraphics[width=0.45\textwidth]{figures/hww_analysis17_125_ALL_incl_0j_mll.pdf}
\end{tabular} 
\caption{ \mT(left) and \mll(right) distributionss for \mHi=125~\GeV\ analysis 
in 0-jet category. 
Top is for \DF, middle is for \SF, and the bottom is for the inclusive channel.  
The data shows good agreement bewteen prediction assmuing \mHi=125~\GeV\ Higgs boson.} 
\label{fig:cutbased125_0jet} 
\end{figure} 
%
\begin{figure}[htp] 
\centering 
\begin{tabular}{c} 
\includegraphics[width=0.45\textwidth]{figures/hww_analysis17_125_ALL_of_1j_mt.pdf}
\includegraphics[width=0.45\textwidth]{figures/hww_analysis17_125_ALL_of_1j_mll.pdf} 
\\
\includegraphics[width=0.45\textwidth]{figures/hww_analysis17_125_ALL_sf_1j_mt.pdf}
\includegraphics[width=0.45\textwidth]{figures/hww_analysis17_125_ALL_sf_1j_mll.pdf}
\\
\includegraphics[width=0.45\textwidth]{figures/hww_analysis17_125_ALL_incl_1j_mt.pdf}
\includegraphics[width=0.45\textwidth]{figures/hww_analysis17_125_ALL_incl_1j_mll.pdf}
\end{tabular} 
\caption{ \mT(left) and \mll(right) distributionss for \mHi=125~\GeV\ analysis 
in 1-jet category. 
Top is for \DF, middle is for \SF, and the bottom is for the inclusive channel.  
The data shows good agreement bewteen prediction assmuing \mHi=125~\GeV\ Higgs boson.} 
\label{fig:cutbased125_1jet} 
\end{figure} 

% 7 TeV
\begin{sidewaystable}
{
 \tiny
  \begin{center}
   \begin{tabular}{l | c c | c c c c c c c c c c | c | c}
    \hline
     process & qqH & ggH & qqWW & ggWW & VV & Top & Zjets & WjetsE & Wgamma & Wg3l & Ztt & WjetsM & $\sum$Bkg & Data \\
      \hline 
      \multicolumn{15}{c}{\mHi = 124~\GeV} \\
      \hline 
      \DF\ 0-jet & $0.2\pm0.0$ & $17.8\pm3.9$ & $68.0\pm7.3$ & $3.3\pm1.1$ & $1.5\pm0.2$ & $3.9\pm0.9$ & $0.5\pm0.4$ & $6.5\pm2.5$ & $3.5\pm2.7$ & $2.9\pm1.3$ & $0.0\pm0.0$ & $4.2\pm1.8$ & $94.3\pm8.6$ & 99 \\ % of 0j
      \DF\ 1-jet & $0.7\pm0.1$ & $6.3\pm2.1$ & $18.1\pm3.2$ & $1.0\pm0.4$ & $1.7\pm0.2$ & $13.1\pm1.0$ & $0.6\pm0.4$ & $3.1\pm1.3$ & $1.0\pm1.0$ & $1.0\pm0.6$ & $0.0\pm0.0$ & $3.0\pm1.4$ & $42.7\pm4.1$ & 48 \\ % of 1j
      \SF\ 0-jet & $0.1\pm0.0$ & $8.1\pm1.8$ & $40.9\pm4.5$ & $1.7\pm0.5$ & $0.8\pm0.1$ & $1.8\pm0.5$ & $10.5\pm4.1$ & $2.0\pm0.8$ & $0.0\pm0.0$ & $0.9\pm0.5$ & $0.0\pm0.0$ & $1.1\pm0.7$ & $59.7\pm6.3$ & 60 \\ %sf 0j
      \SF\ 1-jet & $0.3\pm0.0$ & $2.3\pm0.8$ & $8.7\pm1.6$ & $0.6\pm0.2$ & $0.6\pm0.1$ & $5.7\pm0.5$ & $9.7\pm4.0$ & $0.6\pm0.3$ & $0.0\pm0.0$ & $0.3\pm0.3$ & $0.0\pm0.0$ & $0.1\pm0.4$ & $26.4\pm4.4$ & 29 \\ %sf 1j
    \hline 
      \multicolumn{15}{c}{\mHi = 160~\GeV} \\
    \hline  
    \DF\ 0-jet & $0.9\pm0.1$ & $73.4\pm16.0$ & $40.5\pm4.4$ & $4.4\pm1.4$ & $0.7\pm0.1$ & $4.1\pm1.0$ & $0.0\pm0.0$ & $1.3\pm0.6$ & $0.9\pm1.0$ & $0.3\pm0.2$ & $0.0\pm0.0$ & $1.2\pm0.8$ & $53.4\pm5.0$ & 59 \\
    \DF\ 1-jet & $4.0\pm0.4$ & $33.1\pm10.5$ & $16.7\pm3.0$ & $1.5\pm0.5$ & $1.0\pm0.1$ & $14.1\pm1.1$ & $0.1\pm0.0$ & $1.2\pm0.6$ & $0.0\pm0.0$ & $0.3\pm0.2$ & $0.0\pm0.0$ & $0.4\pm0.5$ & $35.3\pm3.3$ & 32 \\
    \SF\ 0-jet & $0.6\pm0.1$ & $59.3\pm12.9$ & $34.4\pm3.8$ & $3.2\pm1.0$ & $0.5\pm0.1$ & $3.3\pm0.8$ & $3.4\pm3.5$ & $0.4\pm0.3$ & $0.0\pm0.0$ & $0.6\pm0.4$ & $0.0\pm0.0$ & $0.1\pm0.4$ & $45.9\pm5.4$ & 50 \\
    \SF\ 1-jet & $2.8\pm0.3$ & $24.3\pm7.7$ & $12.0\pm2.2$ & $1.2\pm0.4$ & $0.5\pm0.1$ & $10.2\pm0.9$ & $9.0\pm3.8$ & $0.2\pm0.2$ & $0.0\pm0.0$ & $0.1\pm0.2$ & $0.0\pm0.0$ & $1.4\pm0.9$ & $34.7\pm4.6$ & 47 \\
    \hline 
      \multicolumn{15}{c}{\mHi = 200~\GeV} \\
    \hline  
    \DF\ 0-jet & $0.4\pm0.0$ & $28.3\pm6.4$ & $64.3\pm7.1$ & $7.1\pm2.2$ & $1.0\pm0.1$ & $11.1\pm2.5$ & $0.1\pm0.0$ & $2.5\pm1.1$ & $0.0\pm0.0$ & $0.1\pm0.2$ & $0.0\pm0.0$ & $0.3\pm0.5$ & $86.7\pm7.9$ & 85 \\
    \DF\ 1-jet & $2.3\pm0.2$ & $13.7\pm4.1$ & $28.7\pm5.2$ & $2.7\pm0.9$ & $1.1\pm0.1$ & $31.1\pm2.2$ & $0.2\pm0.0$ & $1.8\pm0.8$ & $0.0\pm0.0$ & $0.0\pm0.0$ & $0.0\pm0.0$ & $0.4\pm0.6$ & $66.0\pm5.8$ & 49 \\
    \SF\ 0-jet & $0.4\pm0.0$ & $24.0\pm5.4$ & $49.8\pm5.5$ & $5.8\pm1.8$ & $0.7\pm0.1$ & $6.9\pm1.6$ & $3.8\pm2.6$ & $0.7\pm0.2$ & $0.0\pm0.0$ & $0.0\pm0.0$ & $0.0\pm0.0$ & $0.6\pm0.7$ & $68.3\pm6.6$ & 70 \\
    \SF\ 1-jet & $1.5\pm0.2$ & $9.8\pm3.0$ & $20.2\pm3.7$ & $2.0\pm0.7$ & $0.7\pm0.1$ & $20.6\pm1.5$ & $15.2\pm5.1$ & $0.4\pm0.3$ & $0.0\pm0.0$ & $0.0\pm0.0$ & $0.0\pm0.0$ & $1.7\pm1.0$ & $60.9\pm6.6$ & 56 \\
    \hline
      \multicolumn{15}{c}{\mHi = 400~\GeV} \\
    \hline  
    \DF\ 0-jet & $0.2\pm0.0$ & $11.0\pm3.0$ & $34.6\pm3.8$ & $2.6\pm0.8$ & $0.9\pm0.1$ & $17.4\pm3.9$ & $0.1\pm0.0$ & $3.1\pm1.2$ & $0.0\pm0.0$ & $1.4\pm0.7$ & $0.0\pm0.0$ & $0.2\pm0.5$ & $60.2\pm5.7$ & 58 \\
    \DF\ 1-jet & $0.7\pm0.1$ & $7.6\pm2.3$ & $28.8\pm4.6$ & $1.5\pm0.5$ & $1.1\pm0.1$ & $34.2\pm2.4$ & $1.0\pm0.7$ & $3.5\pm1.4$ & $0.0\pm0.0$ & $0.3\pm0.3$ & $0.0\pm0.0$ & $0.8\pm0.7$ & $71.2\pm5.5$ & 60 \\
    \SF\ 0-jet & $0.1\pm0.0$ & $8.8\pm2.4$ & $25.6\pm2.9$ & $2.0\pm0.6$ & $0.5\pm0.1$ & $11.1\pm2.5$ & $3.0\pm0.3$ & $2.2\pm0.9$ & $0.0\pm0.0$ & $6.1\pm2.7$ & $0.0\pm0.0$ & $0.2\pm0.4$ & $50.8\pm4.8$ & 45 \\
    \SF\ 1-jet & $0.5\pm0.1$ & $5.3\pm1.6$ & $16.3\pm2.6$ & $0.9\pm0.3$ & $0.6\pm0.1$ & $19.8\pm1.4$ & $6.6\pm2.7$ & $1.6\pm0.7$ & $0.0\pm0.0$ & $1.6\pm0.8$ & $0.0\pm0.0$ & $0.1\pm0.4$ & $47.5\pm4.2$ & 65 \\
    \hline 
      \multicolumn{15}{c}{\mHi = 600~\GeV} \\
    \hline  
    \DF\ 0-jet & $0.1\pm0.0$ & $1.7\pm0.6$ & $10.1\pm1.2$ & $0.8\pm0.3$ & $0.2\pm0.0$ & $5.3\pm1.2$ & $0.0\pm0.0$ & $1.1\pm0.5$ & $0.0\pm0.0$ & $0.3\pm0.1$ & $0.0\pm0.0$ & $0.0\pm0.0$ & $17.9\pm1.8$ & 16 \\
    \DF\ 1-jet & $0.3\pm0.0$ & $1.5\pm0.5$ & $10.1\pm1.7$ & $0.5\pm0.2$ & $0.4\pm0.1$ & $9.8\pm0.8$ & $1.0\pm0.7$ & $1.7\pm0.7$ & $0.0\pm0.0$ & $0.2\pm0.2$ & $0.0\pm0.0$ & $0.6\pm0.5$ & $24.1\pm2.2$ & 19 \\
    \SF\ 0-jet & $0.1\pm0.0$ & $1.2\pm0.5$ & $5.5\pm0.7$ & $0.6\pm0.2$ & $0.2\pm0.0$ & $3.3\pm0.8$ & $1.0\pm0.1$ & $0.9\pm0.4$ & $0.0\pm0.0$ & $0.0\pm0.0$ & $0.0\pm0.0$ & $0.2\pm0.3$ & $11.6\pm1.2$ & 13 \\
    \SF\ 1-jet & $0.2\pm0.0$ & $1.0\pm0.3$ & $5.2\pm0.9$ & $0.3\pm0.1$ & $0.2\pm0.0$ & $4.9\pm0.5$ & $0.6\pm0.1$ & $0.9\pm0.4$ & $0.0\pm0.0$ & $0.3\pm0.3$ & $0.0\pm0.0$ & $0.1\pm0.3$ & $12.6\pm1.2$ & 16 \\
   \hline
   \end{tabular}
   \end{center}
    }
    \caption{Table of yields in cut-based analysis at 7~\TeV\ in \intlumiSevenTeV. 
    All final states are shown separately. Yields for each process 
    and the corresponding uncertainties(stats.+syst.) are shown. The last two 
    colunms show the sum of all backgrounds and the data counts. For 7 \TeV, \mHi=124~\GeV\ 
    is used instead of 125~\GeV because 125~\GeV\ sample does not exist. }
    \label{tab:cut7tev}
\end{sidewaystable}

% 8 TeV
\begin{sidewaystable}
{
 \tiny
  \begin{center}
   \begin{tabular}{l | c c | c c c c c c c c c c | c | c}
    \hline
     process & qqH & ggH & qqWW & ggWW & VV & Top & Zjets & WjetsE & Wgamma & Wg3l & Ztt & WjetsM & $\sum$Bkg & Data \\
      \hline 
      \multicolumn{15}{c}{\mHi = 124~\GeV} \\
      \hline 
    \DF\ 0-jet & $1.0\pm0.1$ & $88.9\pm19.3$ & $294.1\pm28.4$ & $16.0\pm5.0$ & $10.2\pm1.0$ & $20.0\pm4.3$ & $0.0\pm0.0$ & $26.4\pm9.6$ & $21.2\pm9.8$ & $18.3\pm8.1$ & $1.2\pm0.2$ & $21.4\pm8.0$ & $428.8\pm34.2$ & 505 \\
    \DF\ 1-jet & $4.6\pm0.5$ & $37.5\pm12.2$ & $74.5\pm10.4$ & $4.8\pm1.6$ & $10.2\pm1.1$ & $78.8\pm4.5$ & $0.0\pm0.0$ & $12.7\pm4.7$ & $6.8\pm4.0$ & $4.5\pm2.3$ & $2.7\pm0.4$ & $12.9\pm5.0$ & $208.0\pm14.1$ & 228 \\
    \SF\ 0-jet & $0.5\pm0.1$ & $55.8\pm12.2$ & $197.2\pm19.2$ & $9.7\pm3.1$ & $13.3\pm1.3$ & $9.3\pm2.2$ & $92.2\pm31.0$ & $12.1\pm4.5$ & $3.2\pm2.5$ & $6.1\pm2.9$ & $0.0\pm0.0$ & $16.5\pm6.2$ & $359.6\pm37.6$ & 421 \\
    \SF\ 1-jet & $2.1\pm0.2$ & $15.9\pm5.2$ & $37.5\pm5.3$ & $2.2\pm0.8$ & $6.5\pm1.0$ & $40.4\pm3.1$ & $14.7\pm5.3$ & $2.8\pm1.2$ & $2.5\pm1.5$ & $0.8\pm0.7$ & $0.0\pm0.0$ & $3.7\pm1.6$ & $111.0\pm8.6$ & 140 \\
    \hline 
      \multicolumn{15}{c}{\mHi = 160~\GeV} \\
    \hline  
    \DF\ 0-jet & $6.2\pm0.6$ & $371.6\pm81.1$ & $175.6\pm17.1$ & $20.7\pm6.5$ & $5.8\pm0.6$ & $24.9\pm5.5$ & $0.0\pm0.0$ & $4.7\pm1.8$ & $4.6\pm3.3$ & $1.7\pm1.1$ & $0.1\pm0.1$ & $1.1\pm0.9$ & $239.3\pm19.5$ & 285 \\
    \DF\ 1-jet & $24.4\pm2.6$ & $181.2\pm57.6$ & $66.1\pm9.3$ & $6.6\pm2.2$ & $7.5\pm0.9$ & $82.9\pm4.7$ & $0.0\pm0.0$ & $6.4\pm2.4$ & $0.2\pm0.2$ & $0.9\pm0.7$ & $0.5\pm0.2$ & $2.3\pm1.4$ & $173.3\pm11.0$ & 226 \\
    \SF\ 0-jet & $4.7\pm0.5$ & $321.2\pm70.1$ & $148.4\pm14.5$ & $15.9\pm5.0$ & $9.7\pm1.0$ & $14.0\pm3.2$ & $18.8\pm9.7$ & $2.5\pm1.1$ & $1.0\pm0.7$ & $0.8\pm0.5$ & $0.0\pm0.0$ & $3.2\pm1.5$ & $214.3\pm18.6$ & 256 \\
    \SF\ 1-jet & $13.7\pm1.5$ & $103.6\pm33.0$ & $42.6\pm6.1$ & $3.7\pm1.3$ & $5.4\pm0.8$ & $47.6\pm3.2$ & $8.8\pm4.0$ & $1.4\pm0.7$ & $1.3\pm1.0$ & $0.0\pm0.0$ & $0.0\pm0.0$ & $2.7\pm1.4$ & $113.6\pm8.3$ & 134 \\
    \hline 
      \multicolumn{15}{c}{\mHi = 200~\GeV} \\
    \hline  
    \DF\ 0-jet & $3.1\pm0.3$ & $150.5\pm34.1$ & $286.2\pm27.8$ & $32.3\pm10.0$ & $10.1\pm1.1$ & $54.6\pm11.2$ & $0.0\pm0.0$ & $5.7\pm2.2$ & $2.1\pm2.2$ & $1.7\pm1.3$ & $0.6\pm0.2$ & $1.3\pm1.0$ & $394.6\pm31.8$ & 471 \\
    \DF\ 1-jet & $13.1\pm1.4$ & $78.1\pm23.7$ & $117.5\pm16.5$ & $11.1\pm3.7$ & $10.4\pm1.2$ & $187.0\pm8.3$ & $0.0\pm0.0$ & $9.4\pm3.5$ & $2.6\pm2.4$ & $0.3\pm0.3$ & $1.4\pm0.3$ & $4.3\pm2.0$ & $344.0\pm19.4$ & 421 \\
    \SF\ 0-jet & $2.6\pm0.3$ & $120.5\pm27.3$ & $230.9\pm22.5$ & $29.0\pm9.0$ & $19.5\pm1.9$ & $41.8\pm8.7$ & $20.1\pm7.7$ & $4.5\pm1.8$ & $1.8\pm1.1$ & $1.1\pm0.8$ & $0.0\pm0.0$ & $2.5\pm1.4$ & $351.1\pm27.1$ & 390 \\
    \SF\ 1-jet & $7.4\pm0.8$ & $47.9\pm14.6$ & $77.6\pm11.0$ & $7.4\pm2.5$ & $11.1\pm1.5$ & $120.1\pm6.2$ & $16.6\pm5.7$ & $2.7\pm1.1$ & $0.0\pm 0.0$ & $0.0\pm0.0$ & $0.0\pm0.0$ & $3.5\pm1.7$ & $239.0\pm 0.0$ & 261 \\
    \hline
      \multicolumn{15}{c}{\mHi = 400~\GeV} \\
    \hline  
    \DF\ 0-jet & $1.2\pm0.1$ & $63.0\pm17.3$ & $195.6\pm23.6$ & $14.0\pm4.5$ & $9.7\pm1.1$ & $91.9\pm18.3$ & $0.0\pm0.0$ & $9.4\pm3.6$ & $2.5\pm2.6$ & $2.7\pm1.7$ & $0.2\pm0.1$ & $0.0\pm0.0$ & $326.2\pm30.6$ & 306 \\
    \DF\ 1-jet & $4.4\pm0.5$ & $42.5\pm12.9$ & $125.1\pm19.7$ & $7.1\pm2.4$ & $10.9\pm1.2$ & $212.6\pm9.0$ & $0.0\pm0.0$ & $14.4\pm5.3$ & $0.4\pm0.4$ & $1.0\pm0.8$ & $2.0\pm0.4$ & $3.2\pm1.6$ & $376.7\pm22.5$ & 361 \\
    \SF\ 0-jet & $1.0\pm0.1$ & $53.9\pm14.8$ & $171.5\pm20.8$ & $10.8\pm3.5$ & $21.3\pm2.1$ & $71.8\pm14.4$ & $30.2\pm23.4$ & $7.4\pm2.8$ & $0.5\pm0.5$ & $0.8\pm0.7$ & $0.0\pm0.0$ & $0.0\pm0.0$ & $314.2\pm34.8$ & 290 \\
    \SF\ 1-jet & $3.0\pm0.3$ & $29.7\pm9.0$ & $69.8\pm11.1$ & $4.1\pm1.4$ & $11.4\pm1.3$ & $122.3\pm6.0$ & $21.2\pm16.5$ & $3.3\pm1.4$ & $0.0\pm 0.0$ & $1.1\pm0.9$ & $0.0\pm0.0$ & $2.1\pm1.2$ & $235.3\pm 0.0$ & 215 \\
    \hline 
      \multicolumn{15}{c}{\mHi = 600~\GeV} \\
    \hline  
    \DF\ 0-jet & $0.6\pm0.1$ & $10.1\pm3.7$ & $61.2\pm7.6$ & $5.1\pm1.7$ & $4.0\pm0.5$ & $30.0\pm6.4$ & $0.0\pm0.0$ & $3.4\pm1.4$ & $2.6\pm2.6$ & $1.3\pm0.9$ & $0.1\pm0.1$ & $0.0\pm0.0$ & $107.7\pm10.6$ & 95 \\
    \DF\ 1-jet & $2.1\pm0.2$ & $8.9\pm2.8$ & $47.1\pm7.5$ & $2.6\pm0.9$ & $4.7\pm0.6$ & $65.0\pm4.1$ & $0.0\pm0.0$ & $7.6\pm2.9$ & $0.0\pm 0.0$ & $0.4\pm0.5$ & $0.8\pm0.2$ & $1.5\pm0.9$ & $129.7\pm 0.0$ & 110 \\
    \SF\ 0-jet & $0.5\pm0.1$ & $8.8\pm3.2$ & $55.7\pm6.9$ & $4.4\pm1.5$ & $7.5\pm0.8$ & $21.9\pm4.8$ & $0.0\pm0.0$ & $2.7\pm1.2$ & $0.5\pm0.5$ & $0.0\pm0.0$ & $0.0\pm0.0$ & $0.0\pm0.0$ & $92.8\pm8.7$ & 94 \\
    \SF\ 1-jet & $1.3\pm0.1$ & $5.7\pm1.8$ & $24.6\pm4.0$ & $1.7\pm0.6$ & $4.2\pm0.6$ & $30.9\pm2.2$ & $0.0\pm0.0$ & $1.6\pm0.7$ & $0.0\pm 0.0$ & $0.0\pm0.0$ & $0.0\pm0.0$ & $0.2\pm0.4$ & $63.2\pm 0.0$ & 63 \\
   \hline
   \end{tabular}
   \end{center}
    }
    \caption{Table of yields in cut-based analysis at 8~\TeV\ in \intlumiEightTeV. 
    All final states are shown separately. Yields for each process 
    and the corresponding uncertainties(stats.+syst.) are shown. The last two 
    colunms show the sum of all backgrounds and the data counts. For 7 \TeV, \mHi=124~\GeV\ 
    is used instead of 125~\GeV because 125~\GeV\ sample does not exist. }
    \label{tab:cut8tev}
\end{sidewaystable}


\section{Shape-based analysis results}  

Fig.~\ref{fig:post2D} shows the 2-dimensional templates for post-fit signal, 
data subtracted by post-fit backgrounds in 0-jet and 1-jet final states. 
The plots show only region where signal is populated defined by 
$60<\mt<120~\GeV$ and $12<\mll<100~\GeV$.  
The data - background plots in 0-jet show good agreement with the \mHi=125~\GeV\
plot. The 0-jet plots do not have enough data to draw any conclusions.  

Fig.~\ref{fig:post1Dprojection_st} shows the stack \mT(top) and \mll(bottom) distributions 
using post-fit results of shape-based analysis in \DF\ final states combining 7 and 8~\TeV.
In combining channels, we try to mimic how likelihood works, 
\textit{i.e.} to give more weights to more sensitive channels, by weighting each bin of 2D 
templates with S/(S+B) of the corresponding bin. The final normalization is done such 
that the signal has the same yields as before reweighting.
The uncertainty band is obtained generating pseudo-data sets constructed by post-fit nuisances. 
Fig.~\ref{fig:post1Dprojection_dmb} shows signal and data subtracted by backgrounds 
using the result in fig.~\ref{fig:post1Dprojection_st}. Both \mT\ and \mll\ plots 
show good agreements between data and SM Higgs at 125~\GeV.  



% 2D
\begin{figure}[htp] 
\centering 
\begin{tabular}{c} 
\includegraphics[width=0.45\textwidth]{figures/2d_postfit_0j_125_sig_paper.pdf}          
\includegraphics[width=0.45\textwidth]{figures/2d_postfit_0j_125_dataminusbkg_paper.pdf}          
\\
\includegraphics[width=0.45\textwidth]{figures/2d_postfit_1j_125_sig_paper.pdf}          
\includegraphics[width=0.45\textwidth]{figures/2d_postfit_1j_125_dataminusbkg_paper.pdf}          
\end{tabular} 
\caption{2D templates of post-fit signal(lest) and data subtracted by post backgrounds(right). 
The plots show only signal region defined by $60<\mt<120~\GeV$ and $12<\mll<100~\GeV$.
In 0-jet category(top) signal and data show a good agreement.  
In 1-jet category(top), data is not enough to draw a definitive conclusion. }
\label{fig:post2D} 
\end{figure} 


% 1D projection : stack
\begin{figure}[htp] 
\centering 
\begin{tabular}{c} 
\includegraphics[width=0.7\textwidth]{figures/st_mt.pdf}
\\
\includegraphics[width=0.7\textwidth]{figures/st_mll.pdf}
\end{tabular} 
\caption{Stacked \mT(top) and \mll(bottom) distributions using post-fit results 
of shape-based analysis
in \DF\ final states combining 7 and 8~\TeV.
In order to give more weights to sensitivie channels, each bin of 2D 
template is weighted by S/(S+B) and the total yield is normalized using the signal yield. 
Both normalization and shape of data show a very good agreement with SM Higgs at 125~\GeV.
The uncertainty is post-fit uncertainty.} 
\label{fig:post1Dprojection_st} 
\end{figure} 
%
\begin{figure}[htp] 
\centering 
\begin{tabular}{c} 
\includegraphics[width=0.7\textwidth]{figures/dataminusbkg_mt.pdf} 
\\
\includegraphics[width=0.7\textwidth]{figures/dataminusbkg_mll.pdf} 
\end{tabular} 
\caption{\mT(top) and \mll(bottom) distributions using post-fit results 
of shape-based analysis in \DF\ final states combining 7 and 8~\TeV.
The signal and data subtracted by backgrounds are compared. 
In order to give more weights to sensitivie channels, each bin of 2D 
template is weighted by S/(S+B) and the total yield is normalized using the signal yield. 
Both normalization and shape of data show a very good agreement with SM Higgs at 125~\GeV. } 
\label{fig:post1Dprojection_dmb} 
\end{figure} 

%%%%%%
\section{Exclusion limit of SM Higgs boson}  

Following the procedure described in section~\ref{sec:stat_exclusion}, 
we calculate the 95 \% \CLs\ limit on the signal strength, 
the ratio of observed signal yield to the expected signal yield 
at a given Higgs mass. The expected median limit and its $1\sigma$/$2\sigma$ uncertainty
bands are shown in yellow and green, respectively, along with the observed limit. 

Fig.~\ref{fig:limit78} shows the exclusion limits of SM Higgs boson combining 
all channels in 7 and 8~\TeV. 
The top is the result of cut-based analysis in all channels.
The observed(expected) exclusion of SM boson at \CLs = 95 \% is 
132-212,310-550~\GeV(120-480~\GeV). 
The bottom is the result of the cut-based analysis in \SF\ channels 
and the shape-based analysis in \DF\ channels.
The observed(expected) exclusion of SM boson at \CLs = 95 \% is 128-600~\GeV(115-575~\GeV). 

Fig.~\ref{fig:limit78_secondhiggs} shows
the exclusion limit of the second SM-Higgs-like boson considering SM Higgs as a backgound.
The cut-based analysis is used in \SF\ channels and the shape-based analysis is used in 
\DF\ channels.
The observed(expected) exclusion of the second SM-Higgs-like boson at \CLs = 95 \% 
is 118-600~\GeV(115-600~\GeV). 

\begin{figure}[htp] 
\centering 
\begin{tabular}{c} 
\includegraphics[width=0.9\textwidth]{figures/table_limits_nj_cut_78TeV_log.pdf} \\
\includegraphics[width=0.9\textwidth]{figures/table_limits_nj_78TeV_log.pdf} 
\end{tabular} 
\caption{Exclusion limits of SM Higgs boson combining all channels in 7 and 8~\TeV. 
The top is the result of cut-based analysis in all channels.
The observed(expected) exclusion of SM boson at \CLs = 95 \% is 132-212~\GeV(120-480~\GeV). 
The bottom is the result of the cut-based analysis in \SF\ channels 
and the shape-based analysis in \DF\ channels.
The observed(expected) exclusion of SM boson at \CLs = 95 \% is 128-600~\GeV(115-575~\GeV).} 
\label{fig:limit78} 
\end{figure} 

\begin{figure}[htp] 
\centering 
\begin{tabular}{c} 
\includegraphics[width=0.9\textwidth]{figures/ana_Moriond13_2D_SMH_7p8TeV_bdt_from110to600_logx1_logy1.pdf} 
\end{tabular} 
\caption{Exclusion limit of the second SM-Higgs-like boson considering SM Higgs as a backgound.
The cut-based analysis is used in \SF\ channels and the shape-based analysis is used in 
\DF\ channels.
The observed(expected) exclusion of the second SM-Higgs-like boson at \CLs = 95 \% 
is 118-600~\GeV(115-600~\GeV).} 
\label{fig:limit78_secondhiggs} 
\end{figure} 




\section{Significance}

Following the procedure described in section~\ref{sec:stat_significance}, 
we calculate the compatibility of data with background-only hypothesis. 
The measure is expressed as significance. 
The expected and observed significance are shown. 

Tabs.~\ref{tab:significance_7tev} - \ref{tab:significance_78tev}  
show observed and expected significances at some selected \mHi, 
for 7~\TeV, 8~\TeV and combination of 7 and 8~\TeV.
The cut-based analysis is used in \SF\ channels 
and the shape-based analysis is used in \DF\ channels.
At \mHi=125~\GeV, the observed/expected significance
is $4.0\sigma/5.1\sigma$ when all channels are combined.

Fig.~\ref{fig:significane_mH} shows the observed and expected 
significance for low Higgs mass hypotheses($\mHi\le200~\GeV$). 
The solid black line represents the significance assuming \mHi = 125~\GeV\ signal. 
The green and yellow bands respresent $1\sigma$/$2\sigma$ uncertainty 
estimated by pseudo data. The dotted line represents the significance 
assuming existence of SM Higgs at the given mass. 
The observed data is within $1\sigma$ of expected significance 
assuming existence of SM Higgs boson at \mHi = 125~\GeV.

Tab.~\ref{tab:sig_diffgenerator} shows the observed and expected significances 
using different generators for the \qqww\ process. Alternative generators, 
MC@NLO and Powheg, were used replacing the default generator, Madgraph.
The result shows that the significance is insensitive to the choice 
of default generator.  


%%%%%%%%
\begin{table}[!htbp]
\begin{center}
\begin{tabular}{c | c c | c c }
\hline \hline 
                 &  \multicolumn{2}{c|}{2D} & \multicolumn{2}{c}{Cut-based} \\
\hline
Higgs Mass(\GeV) & Observed & Expected & Observed & Expected  \\
\hline \hline
%110 & 2.6 & 0.6 & 0.5 & 0.4 \\
%115 & 2.4 & 1.1 & 0.3 & 0.8 \\
%120 & 2.3 & 1.7 & 0.4 & 1.2 \\
125 & 2.3 & 2.5 & 0.8 & 1.7 \\
%135 & 1.9 & 4.5 & 1.0 & 3.1 \\
%140 & 1.5 & 5.5 & 0.5 & 3.8 \\
%150 & 1.2 & 7.2 & 0.8 & 5.0 \\
160 & 0.9 & 10.4 & 0.0 & 8.2 \\
%170 & 0.0 & 9.8 & 0.0 & 7.9 \\
%180 & 0.0 & 7.2 & 0.0 & 5.9 \\
%190 & 0.0 & 5.3 & 0.0 & 4.0 \\
200 & 0.0 & 3.7 & 0.0 & 2.9 \\
%250 & 0.0 & 2.0 & 0.0 & 1.5 \\
%300 & 1.2 & 1.7 & 0.0 & 1.4 \\
%350 & 0.3 & 2.1 & 0.0 & 1.6 \\
400 & 0.2 & 1.9 & 0.0 & 1.5 \\
%450 & 0.0 & 1.6 & 0.0 & 1.4 \\
%500 & 0.2 & 1.3 & 0.0 & 1.1 \\
%550 & 0.9 & 1.1 & 0.0 & 0.9 \\
600 & 0.0 & 0.9 & 0.0 & 0.8 \\
\hline \hline
\end{tabular}
\caption{Observed and expected significances in 7~\TeV.   
Cut-based analysis is used in \SF\ final states 
and shape-based analysis is used in \DF\ finale states} 
\label{tab:significance_7tev}
\end{center}
\end{table} 

%%%%%%%%
\begin{table}[!htbp]
\begin{center}
\begin{tabular}{c | c c | c c }
\hline \hline 
                 &  \multicolumn{2}{c|}{2D} & \multicolumn{2}{c}{Cut-based} \\
\hline
Higgs Mass(\GeV) & Observed & Expected & Observed & Expected  \\
\hline \hline
%110 & 2.9 & 0.9 & 2.3 & 0.6 \\
%115 & 3.2 & 1.8 & 2.3 & 1.2 \\
%120 & 3.2 & 3.0 & 1.0 & 1.8 \\
125 & 3.5 & 4.7 & 2.1 & 2.6 \\
%130 & 3.8 & 6.5 & 2.5 & 3.4 \\
%135 & 4.2 & 8.3 & 2.6 & 4.4 \\
%140 & 4.5 & 10.1 & 2.5 & 5.2 \\
%150 & 4.3 & 14.1 & 2.8 & 7.4 \\
160 & 4.1 & 20.5 & 0.0 & 11.5 \\
%170 & 3.5 & 17.5 & 0.0 & 10.7 \\
%180 & 3.1 & 12.4 & 2.5 & 8.4 \\
%190 & 2.5 & 8.6 & 2.6 & 5.6 \\
200 & 1.4 & 6.9 & 2.5 & 4.3 \\
%250 & 0.0 & 3.8 & 1.2 & 2.0 \\
%300 & 0.0 & 3.5 & 0.2 & 1.9 \\
%350 & 0.0 & 4.0 & 0.0 & 2.2 \\
400 & 0.0 & 3.8 & 0.0 & 2.1 \\
%450 & 0.0 & 3.1 & 0.0 & 2.0 \\
%500 & 0.0 & 2.4 & 0.0 & 1.7 \\
%550 & 0.0 & 2.0 & 0.0 & 1.4 \\
600 & 0.0 & 1.8 & 0.0 & 1.3 \\
\hline \hline
\end{tabular}
\caption{Observed and expected significances in 8~\TeV.   
Cut-based analysis is used in \SF\ final states 
and shape-based analysis is used in \DF\ finale states} 
\label{tab:significance_8tev}
\end{center}
\end{table} 


%%%%%%%%
\begin{table}[!htbp]
\begin{center}
\begin{tabular}{c | c c | c c }
\hline \hline 
                 &  \multicolumn{2}{c|}{2D} & \multicolumn{2}{c}{Cut-based} \\
\hline
Higgs Mass(\GeV) & Observed & Expected & Observed & Expected  \\
\hline \hline
%110 & 3.7 & 1.1 & 2.1 & 0.6 \\
%115 & 3.8 & 2.0 & 2.1 & 1.2 \\
%120 & 3.8 & 3.3 & 1.9 & 1.8 \\
125 & 4.0 & 5.1 & 2.1 & 2.7 \\
%130 & 4.3 & 7.2 & 2.3 & 3.5 \\
%135 & 4.5 & 9.2 & 2.5 & 4.4 \\
%140 & 4.6 & 11.2 & 2.4 & 5.2 \\
%150 & 4.2 & 15.2 & 2.8 & 7.5 \\
160 & 4.0 & 22.0 & 2.8 & 11.6 \\
%170 & 3.2 & 18.9 & 2.1 & 10.7 \\
%180 & 2.7 & 13.1 & 2.4 & 8.4 \\
%190 & 2.0 & 9.1 & 2.4 & 5.6 \\
200 & 1.3 & 7.2 & 2.5 & 4.3 \\
%250 & 0.0 & 4.2 & 0.9 & 2.1 \\
%300 & 0.0 & 3.8 & 0.2 & 2.0 \\
%350 & 0.0 & 4.4 & 0.0 & 2.3 \\
400 & 0.0 & 4.1 & 0.0 & 2.2 \\
%450 & 0.0 & 3.4 & 0.0 & 2.1 \\
%500 & 0.0 & 2.7 & 0.0 & 1.8 \\
%550 & 0.0 & 2.2 & 0.0 & 1.5 \\
600 & 0.0 & 2.0 & 0.0 & 1.4 \\
\hline \hline
\end{tabular}
\caption{Observed and expected significances combining 7~\TeV\ and 8~\TeV\ results.  
Cut-based analysis is used in \SF\ final states 
and shape-based analysis is used in \DF\ finale states} 
\label{tab:significance_78tev}
\end{center}
\end{table} 

%
\begin{figure}[htp] 
\centering 
\begin{tabular}{c} 
\includegraphics[width=0.8\textwidth]{figures/signif_allcomb_inj125_data_zoom.pdf} 
\end{tabular} 
\caption{Observed and expected significance as a function of \mHi\ for 
the low Higgs mass hypotheses($\mHi\le200~\GeV$). 
The solid black line represents the significance assuming \mHi = 125~\GeV\ signal. 
The green and yellow bands respresent $1\sigma$/$2\sigma$ uncertainty 
estimated by pseudo data. The dotted line represents the significance 
assuming existence of SM Higgs at the given mass. 
} 
\label{fig:significane_mH} 
\end{figure} 

%
\begin{table}[htp] 
\begin{center} 
\begin{tabular}{cc|cc|cc} 
\hline 
\multicolumn{2}{c|}{MC@NLO}   &  \multicolumn{2}{c|}{Powheg} & \multicolumn{2}{c}{Madgraph} \\
\hline \hline 
Observed & Expected & Observed & Expected &  Observed & Expected \\ 
\hline 
4.2 & 5.3 & 3.9 & 5.1 & 4.0 & 5.1 \\
\hline 
\end{tabular} 
\caption{The observed and expected significances
using different generators for the \qqww\ process. Alternative generators,
MC@NLO and Powheg, were used replacing the default generator, Madgraph.
The result shows that the significance is insensitive to the choice
of default generator.} 
\label{tab:sig_diffgenerator} 
\end{center} 
\end{table} 


\section{Signal Strength}

As mentioned before, we measure the signal strength at the 
measured Higgs mass. The mass measurement comes from 
$H\rightarrow ZZ\rightarrow4l$ and $H\rightarrow \gamma\gamma$, 
and it turned out to be around 125~\GeV. 
For the results shown in this section,   
the cut-based analysis is used in \SF\ final states 
and shape-based analysis is used in \DF\ finale states. 

Fig.~\ref{fig:mu_mH} shows the best fit signal strength 
as a function of \mHi\ for the low Higgs mass hypotheses($\mHi\le200~\GeV$).
All channels in 7~\TeV\ and 8~\TeV\ are combined. 
The green band shows $1\sigma$ error of the fit. 
The signal strength is within $1\sigma$ of SM Higgs assumption 
in \mHi = 118 - 125~\GeV. 
The variance of the signal strength by changing \mHi\ by 1~\GeV\
is 0.82 and 0.72 for \mHi=126~\GeV\ and \mHi=124~\GeV.  

Fig.~\ref{fig:mu_scan} shows the  $- 2\Delta\ln \textrm{L}$ scan of $\mu$. 
The black curve represents the case where systematic and statistical 
uncertainties are taken into account, while blue curve represents the case 
where only statistical uncertainty is considered. For latter,  
all nuisances are fixed to the post-fit values, and fitted again.  
The uncertainty for the blue curve comes solely from statistics of data. 
The red lines represent the $1\sigma$ unceratinty band for each curve. 
The measured signal strength is $0.76 \pm 0.21$.  
One can extract the contribution of systematic uncertainties to the signal strength
by subtracting uncertainty in blue from the uncertainty in black.
Separating the systematic and statistical uncertainties, 
the measured signal strength is $0.76 \pm 0.16(syst.) \pm 0.13(stat.)$.  

Fig.~\ref{fig:mu_allchannels} shows the fitted signal strength($\mu$) 
for individual channels for \mHi=125~\GeV.
The dotted vertical line and the green band are 
the central value and uncertainty band of the signal strength 
obtained from combination of all channels. 
The figure shows that all channels are consistent each other
within the uncertainties.

Tab.~\ref{tab:mu_diffgenerator} shows the signal strengths 
using different generators for the \qqww\ process. Alternative generators,
MC@NLO and Powheg, were used replacing the default generator, Madgraph.
The result shows that the signal strength is insensitive to the choice
of default generator. 


%
\begin{figure}[htp] 
\centering 
\begin{tabular}{c} 
\includegraphics[width=0.8\textwidth]{figures/mlf7p8TeV_zoomed.pdf}
\end{tabular} 
\caption{The best fit signal strength($\mu$) as a function of \mHi\ for low Higgs 
mass hypotheses($\mHi\le200~\GeV$).
All channels are combined. 
Cut-based analysis is used in \SF\ final states 
and shape-based analysis is used in \DF\ finale states} 
\label{fig:mu_mH} 
\end{figure} 

%
\begin{figure}[htp] 
\centering 
\begin{tabular}{c} 
\includegraphics[width=0.8\textwidth]{figures/MuDeltaNLL.pdf}
\end{tabular} 
\caption{ $- 2\Delta\ln \textrm{L}$ scan of $\mu$ with(black) 
and without(blue) systematic uncertainty.
The uncertainty for the blue curve comes solely from statistics of data.
The red lines represent the $1\sigma$ unceratinty band for each curve. 
One can extract the contribution of systematic uncertainties to the signal strength
by subtracting uncertainty in blue from the uncertainty in black.} 
\label{fig:mu_scan} 
\end{figure} 

%
\begin{figure}[htp] 
\centering 
\begin{tabular}{c} 
\includegraphics[width=0.8\textwidth]{figures/mu_allchannels.pdf}
\end{tabular} 
\caption{ Signal strength($\mu$) for individual channels for \mHi=125~\GeV.
Cut-based analysis is used in \SF\ final states 
and shape-based analysis is used in \DF\ finale states. 
The dotted vertical line and the green band are 
the central value and uncertainty band of the signal strength 
obtained from combination of all channels. 
All channels are consistent to each other.
} 
\label{fig:mu_allchannels} 
\end{figure} 

%
\begin{table}[htp] 
\begin{center} 
\begin{tabular}{c|c|c} 
\hline 
MC@NLO   &  Powheg & Madgraph \\
\hline \hline 
$0.82 \pm 0.24$ & $0.74 \pm 0.21$ & $0.76 \pm 0.21$ \\
\hline 
\end{tabular} 
\caption{ Signal strengths 
using different generators for the \qqww\ process. Alternative generators,
MC@NLO and Powheg, were used replacing the default generator, Madgraph.
The result shows that the significance is insensitive to the choice
of default generator.} 
\label{tab:mu_diffgenerator} 
\end{center} 
\end{table} 



