%
%
% UCSD Doctoral Dissertation Template
% -----------------------------------
% http://ucsd-thesis.googlecode.com
%
%


%% REQUIRED FIELDS -- Replace with the values appropriate to you

% No symbols, formulas, superscripts, or Greek letters are allowed
% in your title.
%\title{Evidence of a new boson that decays to two W bosons 
%in the full leptonic final states in 24.4 $fb^{-1}$ at 
%center-of-mass energy of 7 and 8 TeV with Compact Muon Solenoid detector}
\title{
Evidence of Higgs to WW to Two Leptons and Two Neutrinos at CMS
}
\author{Jae Hyeok Yoo}
\degreeyear{2014}

% Master's Degree theses will NOT be formatted properly with this file.
\degreetitle{Doctor of Philosophy} 

\field{Physics}
\chair{Professor Frank W\"urthwein}
% Uncomment the next line iff you have a Co-Chair
% \cochair{Professor Cochair Semimaster} 
%
% Or, uncomment the next line iff you have two equal Co-Chairs.
%\cochairs{Professor Chair Masterish}{Professor Chair Masterish}

%  The rest of the committee members  must be alphabetized by last name.
\othermembers{
Professor Claudio Campagnari\\
Professor Pamela Cosman \\
Professor Aneesh Manohar \\
Professor Avraham Yagil \\ 
}
\numberofmembers{4} % |chair| + |cochair| + |othermembers|


%% START THE FRONTMATTER
%
\begin{frontmatter}

%% TITLE PAGES
%
%  This command generates the title, copyright, and signature pages.
%
\makefrontmatter 

%% DEDICATION
%
%  You have three choices here:
%    1. Use the ``dedication'' environment. 
%       Put in the text you want, and everything will be formated for 
%       you. You'll get a perfectly respectable dedication page.
%   
%
%    2. Use the ``mydedication'' environment.  If you don't like the
%       formatting of option 1, use this environment and format things
%       however you wish.
%
%    3. If you don't want a dedication, it's not required.
%
%
\begin{dedication} 
\vspace{5cm}
\textit{Soli Deo gloria.}
\end{dedication}


% \begin{mydedication} % You are responsible for formatting here.
%   \vspace{1in}
%   \begin{flushleft}
% 	To me.
%   \end{flushleft}
%   
%   \vspace{2in}
%   \begin{center}
% 	And you.
%   \end{center}
% 
%   \vspace{2in}
%   \begin{flushright}
% 	Which equals us.
%   \end{flushright}
% \end{mydedication}



%% EPIGRAPH
%
%  The same choices that applied to the dedication apply here.
%
%\begin{epigraph} % The style file will position the text for you.
%  \emph{A careful quotation\\
%  conveys brilliance.}\\
%  ---Smarty Pants
%\end{epigraph}

% \begin{myepigraph} % You position the text yourself.
%   \vfil
%   \begin{center}
%     {\bf Think! It ain't illegal yet.}
% 
% 	\emph{---George Clinton}
%   \end{center}
% \end{myepigraph}


%% SETUP THE TABLE OF CONTENTS
%
\tableofcontents
\listoffigures  % Uncomment if you have any figures
\listoftables   % Uncomment if you have any tables



%% ACKNOWLEDGEMENTS
%
%  While technically optional, you probably have someone to thank.
%  Also, a paragraph acknowledging all coauthors and publishers (if
%  you have any) is required in the acknowledgements page and as the
%  last paragraph of text at the end of each respective chapter. See
%  the OGS Formatting Manual for more information.
%
\begin{acknowledgements} 
 First of all, I would like to present my deep gratitude to my advisor, professor Frank W\"urthwein. 
 He always exceeded my expectation as a good advisor by a large margin and I really feel fortunate to 
 have him as my advisor. Needless to say, he is an excellent physicist who has an exceptional 
 talent to explain complicated things in a simple form. In addition, he is a good person who really 
 cares for his students.  

 I want to thank professors Avi Yagil and Claudio Campagnari. Avi trained me to be 
 crispier in communication, particularly, when presenting my work to others.
 Claudio showed me how to draw a big picture when planning a workflow. I was always surprised 
 by his vast knowledge. 
 I really appreciate them for the supports and opportunites they provided to me. 

 I want to thank my thesis committee members for their time and effort to make my thesis better. 
 Professors Frank W\"urthwein, Avi Yagil, Claudio Campagnari, Aneesh Manohar and Pamela Cosman.  
 
 My gratitudes go to the members of Smurf team,  Dmytro(Dima) Kovalskyi, Yanyan Gao, Dave Evans, 
 Giuseppe Cerati, Guillelmo Gomez Ceballos Retuerto and Marco Zanetti, 
 with whom I closely worked for my thesis work(HWW). 
 Dima gave me a lot of opportunities to me so that I can contribute to the analysis. 
 Yanyan and Dave shared their code which I used to do various analyses including the 2D analysis. 
 I shared office with Yanyan during my stay at CERN, and I learned a lot of physics from her, 
 so I call her Shifu(teacher in Chinese). Dave helped me from the first project I worked on in 2010 
 to the thesis work. Whenever I had problems in analysis or something else including how to 
 fix the battery in Frank's car, I went to him and there came solutions.
 I started HWW analysis with Giuseppe working on DY MVA. He understood me when I knew nothing 
 and generously helped me get up to speed. Guillelmo is probably the guy whom I bugged most 
 in the Smurf team because we had to crosscheck results whenever results should go out. 
 He is the most amazing guy I met in CMS because of the amount of work he did and how 
 efficiently he managed those works on his plate. 
 
 My gratitudes go to the members of Surf'n Turf(SNT) group. Frank Golf helped me a lot 
 with not only technical issues but also basic knowledge about particle physics 
 when I was a baby in the group. I am glad that we will be sharing office in Broida Hall 
 changing only one letter in the institution name. Warren Andrew, Jacob Ribnik and Puneeth Kalavase
 helped me with many technical problems in my early years. Yanjun Tu and Sanjay Padhi helped me 
 on my first project in the group, and Yanjun particularly helped me with data processing. 
 My fellow students, Ryan Kelley, Ian MacNaill and Vince Welke, 
 helped me understand American culture and became my English teachers. Ryan and Ian helped 
 me with many coding issues. I have been 
 sharing office with Vince for one and a half years, and it has been a great fun to discuss 
 physics and ``beyond" physics issues such as where the best Korean BBQ place in San Diego is.  
 SNT is a huge group and I got helps from everyone in one way or the other. Please 
 forgive me that I did not spell all out. 

 I want to thank my friends at UCSD and CERN. At UCSD, my fellow physics graduate students,
 Chi Yung Richard Chim, Yaojun Zhang, Ethan Cho and Xiang Zhai, and the members 
 of Korean Soccer Freak(KSF), particularly, Jong Woo Kim who initiated KSF with me 
 and worked in the same building until now. 
 At CERN, Dong Ho Moon, Mi Hee Jo, Geum Bong Yu, Philip Chang, Sang Eun Lee and Hwi Dong Yoo
 with whom I shared valuable memories. Without them my life in US and Europe would have been 
 dry and barren. 

 Last but not least, I want to thank my family. My parents and sister for always trusting 
 my decision and fully supporting whatever I do. Even though they could, they did not 
 demand anything from me, but waited in patience sacrificing many things in their lives. 
 My parent-in-laws for their support and help particularly when my sons were born. 
 My mother-in-law flew over the Pacific Ocean and spent a couple of months to take care 
 of my wife and sons. I feel sorry for not having many opportunites to take her to 
 good places, which she deserved. My brother-in-law for his support and cares. 
 My wife, So Jung Kim, for her endless love and understanding. She had to travel a lot because of 
 me, and so life was always unsettled. But, she understood all these and supported me 
 so that I could concentrate on my work. My two sons, Timothy Jihun Yoo and Elliot Jisuk Yoo, 
 for the joy and the happiness they brought to our home. 
 \\ 

 Chapter~\ref{ch:background_estimation},\ref{ch:fit_validation}, \ref{ch:results},
 \ref{ch:spin} and~\ref{ch:conclusion}, 
 are reprints of the material as it appears in CMS Collaboration, 
 ``Measurement of Higgs boson production and properties in the WW decay channel with 
 leptonic final states", J. High Energy Phys. 01 (2014) 096. 
 The dissertation author was the primary investigator and author of this paper. 

 Chapter~\ref{ch:conclusion},  
 is a reprint of the material as it appears in CMS Collaboration, 
 ``Observation of a new boson at a mass of 125 GeV with the CMS experiment at the LHC", 
 Phys. Lett. B 716, 30 (2012).
 The dissertation author was the primary investigator and author of this paper. 


\end{acknowledgements}


%% VITA
%
%  A brief vita is required in a doctoral thesis. See the OGS
%  Formatting Manual for more information.
%
\begin{vitapage}
\begin{vita}
  \item[2007] B.S. in Physics, Korea University 
  \item[2009] M.S. in Physics, Korea University 
  \item[2014] Ph.D. in Physics, University of California, San Diego 
\end{vita}
\begin{publications}
  \item \textbf{CMS Collaboration}, {``Measurement of Higgs boson production and properties in the WW decay channel with leptonic final states ''}, \emph{J. High Energy Phys.} 01 (2014) 096
  \item \textbf{CMS Collaboration}, {``Observation of a new boson at a mass of 125 GeV with the CMS experiment at the LHC''}, \emph{Phys. Lett. B} 716, 30 (2012) 
\end{publications}
\end{vitapage}


%% ABSTRACT
%
%  Doctoral dissertation abstracts should not exceed 350 words. 
%   The abstract may continue to a second page if necessary.
%
\begin{abstract}
  In this thesis, we report the result on the search for 
  the Standard Model(SM) Higgs boson decaying to a pair of W in the full leptonic final state
  using data collected by CMS detector at LHC at 7 and 8~\TeV. The integrated luminosity 
  is \intlumiSevenTeV\ and \intlumiEightTeV\ at 7 and 8~\TeV, respectively. 
  The SM Higgs hypothesis is excluded at \CLs=95~\%\ in $\mHi=128-600~\GeV$.   
  An excess of data is observed around $\mHi=125~\GeV$ which corresponds to significance 
  of $4.0\sigma$ on the background-only hypothesis. The measured production rate normalized 
  to the SM prediction is $0.76 \pm 0.13(stat.) \pm 0.16(syst.)$. 
  A hypothesis test on the spin-parity nature of the new boson shows that data prefers 
  SM Higgs boson to the graviton-like spin-2 model or spin-0 pseudo-scalar model. 

\end{abstract}


\end{frontmatter}
