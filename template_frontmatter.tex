%
%
% UCSD Doctoral Dissertation Template
% -----------------------------------
% http://ucsd-thesis.googlecode.com
%
%


%% REQUIRED FIELDS -- Replace with the values appropriate to you

% No symbols, formulas, superscripts, or Greek letters are allowed
% in your title.
%\title{Evidence of a new boson that decays to two W bosons 
%in the full leptonic final states in 24.4 $fb^{-1}$ at 
%center-of-mass energy of 7 and 8 TeV with Compact Muon Solenoid detector}
\title{
Evidence of a New Boson that Decays to Two W Bosons 
and a Study of Its Spin-parity Nature 
in the Full Leptonic Final States 
at 7 and 8 TeV Center-of-mass Energy 
with the Compact Muon Solenoid Detector
}
\author{Jae Hyeok Yoo}
\degreeyear{2014}

% Master's Degree theses will NOT be formatted properly with this file.
\degreetitle{Doctor of Philosophy} 

\field{Physics}
\chair{Professor Frank W\"urthwein}
% Uncomment the next line iff you have a Co-Chair
% \cochair{Professor Cochair Semimaster} 
%
% Or, uncomment the next line iff you have two equal Co-Chairs.
%\cochairs{Professor Chair Masterish}{Professor Chair Masterish}

%  The rest of the committee members  must be alphabetized by last name.
\othermembers{
Professor Claudio Campagnari\\
Professor Pamela Cosman \\
Professor Aneesh Manohar \\
Professor Avraham Yagil \\ 
}
\numberofmembers{4} % |chair| + |cochair| + |othermembers|


%% START THE FRONTMATTER
%
\begin{frontmatter}

%% TITLE PAGES
%
%  This command generates the title, copyright, and signature pages.
%
\makefrontmatter 

%% DEDICATION
%
%  You have three choices here:
%    1. Use the ``dedication'' environment. 
%       Put in the text you want, and everything will be formated for 
%       you. You'll get a perfectly respectable dedication page.
%   
%
%    2. Use the ``mydedication'' environment.  If you don't like the
%       formatting of option 1, use this environment and format things
%       however you wish.
%
%    3. If you don't want a dedication, it's not required.
%
%
\begin{dedication} 
\vspace{5cm}
\textit{Soli Deo gloria.}
\end{dedication}


% \begin{mydedication} % You are responsible for formatting here.
%   \vspace{1in}
%   \begin{flushleft}
% 	To me.
%   \end{flushleft}
%   
%   \vspace{2in}
%   \begin{center}
% 	And you.
%   \end{center}
% 
%   \vspace{2in}
%   \begin{flushright}
% 	Which equals us.
%   \end{flushright}
% \end{mydedication}



%% EPIGRAPH
%
%  The same choices that applied to the dedication apply here.
%
%\begin{epigraph} % The style file will position the text for you.
%  \emph{A careful quotation\\
%  conveys brilliance.}\\
%  ---Smarty Pants
%\end{epigraph}

% \begin{myepigraph} % You position the text yourself.
%   \vfil
%   \begin{center}
%     {\bf Think! It ain't illegal yet.}
% 
% 	\emph{---George Clinton}
%   \end{center}
% \end{myepigraph}


%% SETUP THE TABLE OF CONTENTS
%
\tableofcontents
\listoffigures  % Uncomment if you have any figures
\listoftables   % Uncomment if you have any tables



%% ACKNOWLEDGEMENTS
%
%  While technically optional, you probably have someone to thank.
%  Also, a paragraph acknowledging all coauthors and publishers (if
%  you have any) is required in the acknowledgements page and as the
%  last paragraph of text at the end of each respective chapter. See
%  the OGS Formatting Manual for more information.
%
\begin{acknowledgements} 
 Thanks ... 

\end{acknowledgements}


%% VITA
%
%  A brief vita is required in a doctoral thesis. See the OGS
%  Formatting Manual for more information.
%
\begin{vitapage}
\begin{vita}
  \item[2007] B.S. in Physics, Korea University 
  \item[2009] M.S. in Physics, Korea University 
  \item[2014] Ph.D. in Physics, University of California, San Diego 
\end{vita}
\begin{publications}
  \item \textbf{CMS Collaboration}, {``Measurement of Higgs boson production and properties in the WW decay channel with leptonic final states ''}, \emph{J. High Energy Phys.} 01 (2014) 096
  \item \textbf{CMS Collaboration}, {``Observation of a new boson at a mass of 125 GeV with the CMS experiment at the LHC''}, \emph{Phys. Lett. B} 716, 30 (2012) 
\end{publications}
\end{vitapage}


%% ABSTRACT
%
%  Doctoral dissertation abstracts should not exceed 350 words. 
%   The abstract may continue to a second page if necessary.
%
\begin{abstract}
  In this thesis, we report the result on the measurement of the production rate 
  of the Standard Model(SM) Higgs boson decaying to a pair of W in the full leptonic final state
  using data collected by CMS detector at LHC at 7 and 8~\TeV. The integrated luminosity 
  is \intlumiSevenTeV\ and \intlumiEightTeV\ at 7 and 8~\TeV, respectively. 
  The SM Higgs hypothesis is excluded at \CLs=95~\%\ in $\mHi=128-600~\GeV$.   
  An excess of data is observed around $\mHi=125~\GeV$ which corresponds to significance 
  of $4.0\sigma$ on the background-only hypothesis. The measured production rate normalized 
  to the SM prediction is $0.76 \pm 0.16(syst.) \pm 0.13(stat.)$. 
  A hypothesis test on the spin-parity nature of the new boson shows that data prefers 
  SM Higgs boson to the graviton-like spin-2 model or spin-0 pseudo-scalar model. 

\end{abstract}


\end{frontmatter}
