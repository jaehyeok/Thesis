Discuss how basic objects(vertex, electron, muon, jet, MET, top-tagging) 
are reconstructed and what selections are.
Other selections will be discussed in the next chapter, Signal Extraction.

%%%%%%%%%%%%%%%%%%%%%%%%%%%%%%%%%%%%%%%%%%%%%%%%%%%%%%%%%%%%%%%%%%
\section{Trigger}
\label{sec:trigger}
\begin{itemize} 
\item \textcolor{red}{list of triggers} 
\item \textcolor{red}{what are the requirements : already in AN }
\end{itemize} 

As seen in section~\ref{subsec:kinimetic_variables}, \hww{} events have trailing lepton 
whose transverse momentum goes down very low for low \mHi{} hypotheses. Triggering low 
\pt{} leptons is very challenging because of large background events. 
Therefore, in order to record signal events, we need to trigger on the leading lepton, 
or on both leptons. The leading lepton option is not possible because the identification 
and isolation requirements should be very tight and momentum thresholds should be very 
high to maintain sustainable bandwidth. Thus, we trigger on the both leptons. 
The double-lepton triggers we designed for this analysis have high efficiency for  
signal events, but are loose enough to collect events in the several control regions 
we used for various studies. We also use control region triggers that allow 
fake rate and lepton selection efficiency measurements with the precision
good enough for this analysis. 

\textcolor{red}{Where can I check the bandwidth of triggers? want to know 
the total bandwidth of CMS and the bandwidth of triggers we use}

\subsection{Analysis Triggers}

The double-lepton triggers that are listed in Table~\ref{tab:trg_doublelepton} require 
two HLT objects and each of them is required to match an L1 seed. The offline lepton \pt{} 
requirement is 20/10 \GeV, so the online lepton \pt{} reuquirement is a bit looser, 17/8 \GeV, 
in order to be safe from possible tigher online selection. In addition,
the longitudinal distance between the two vertices of the leptons is required 
to be less than 0.2 cm in order to reduce Pile-Up events. 

For electron HLT objects there are additional requirements on
shower shapes, %(H/E, $\sigma_{\eta\eta}$), 
track-to-cluster matching, %($|\Delta\eta|$, $|\Delta\phi|$, $|\frac{1}{E}-\frac{1}{p}|$),
track/calorimeter isolation. %(ECalIso, HCalIso, TrkIso). 
The exact variables and cut values are described in Table~\ref{tab:trg_requirement_def}.
In the table, the naming convention of CMS HLT triggers are shown 
with the corresponding requirements. 
H/E is the ratio of energy deposit in HCAL to that of ECAL. 
$\sigma_{\eta\eta}$ is the weighted sum of \Eta{} difference between the 
seed crystal and the 5x5 crystals surrouding the it.   
$|\Delta\eta|(|\Delta\phi|)$ is the difference in absolute value between 
the center of the supercluster and the direction of the track trajectory 
in \Eta($\phi$) direction.   
$|\frac{1}{E}-\frac{1}{p}|$ is the difference between the reciprocal of supercluster energy 
and the reciprocal of the track momentum.  
$\mathrm{ECalIso/E_T}$, $\mathrm{HCalIso/E_T}$, and $\mathrm{TrkIso/E_T}$ 
are the sum of the trasverse energy within $dR<0.3$ \textcolor{red}{(checked)} 
around the center of energy deposit or track trajectory
divided by the transverse energy, $\mathrm{E_T}$. 
% E_T(SC) for Ecal, and elecand.pt() for Hcal and Tk (guess this is track momentum)  
Because simplified algorithm is used for online variables, the variables 
do not exactly correspond to the offline ones. To account for this, we measure 
tirgger efficiency with respect to the offline selection and do corrections accordingly. 
The details on this can be found in~\ref{subsec:trg_eff} where trigger efficiency 
measurement is discussed.
\textcolor{red}{what is TkMu8? what is the iso requirement on single muon triggers?} 

\begin{table}[!ht]
  \centering 
  \begin{tabular} {l|l}
  \hline
  Double-lepton trigger name & L1 seed \\
  \hline \hline
  HLT\_Ele17\_CaloIdT\_CaloIsoVL\_TrkIdVL\_TrkIsoVL\_ 	    &  L1\_DoubleEG\_13\_7  \\
  Ele8\_CaloIdT\_CaloIsoVL\_TrkIdVL\_TrkIsoVL\_v[15-19] 	&                       \\ 
  %190456-190738 %190762-191419 %191512-194533
  \hline
  HLT\_Mu17\_Mu8\_v[16-22] 	    & L1\_DoubleMu\_10\_Open    \\ %190456-193686 %193806-194533
  HLT\_Mu17\_TkMu8\_v[9-14] 	& OR L1\_DoubleMu\_10\_3p5  \\ %190456-193686 %193806-194533
  \hline
  HLT\_Mu17\_Ele8\_CaloIdT\_CaloIsoVL\_ 	& L1\_Mu12\_EG7     \\
  TrkIdVL\_TrkIsoVL\_v[4-9] 	            &     \\
  %190456-190738 %190762-191419 %191512-193686 %193806-194533
  HLT\_Mu8\_Ele17\_CaloIdT\_CaloIsoVL\_	    & L1\_MuOpen\_EG12       \\ 
  TrkIdVL\_TrkIsoVL\_v[4-9] 	            & OR L1\_Mu3p5\_EG12     \\ 
  %190456-190738 %190762-191419 %191512-193686 %193806-194533
  \hline
  \end{tabular} 
  \caption{Double-lepton triggers used to collect signal events.} 
  \label{tab:trg_doublelepton}
\end{table}
%
\begin{table}[!ht]
  \centering 
  \begin{tabular} {l|l}
  \hline
  Single-lepton trigger name & L1 seed \\
  \hline \hline
  HLT\_Ele27\_WP80\_v[8-11] & L1\_SingleEG20 OR L1\_SingleEG22  \\ 
  %190456-190738 %190762-191419 %191512-194533
  \hline 
  HLT\_IsoMu24\_eta2p1\_v[11-15]   & L1\_SingleMu16er  \\  
  %190456-190738 %190762-193686 %193806-194533
  \hline \hline
  \end{tabular}
  \caption{Single-lepton triggers used to collect signal events.} 
  \label{tab:trg_singlelepton}
\end{table}

%
\begin{table}[!ht]
 \centering
 \begin{tabular}{l|c}
   \hline
   name                       &  criterion \\
   \hline \hline
   \multirow{2}{*}{CaloId\_T} & $\mathrm{H/E < 0.15 (0.10) }$ \\
                               & $\sigma_{\eta\eta}\mathrm{< 0.011\;(0.031)}$ \\
    \hline
   \multirow{2}{*}{CaloId\_VT} & $\mathrm{H/E < 0.05 (0.05) }$ \\
                               & $\sigma_{\eta\eta}\mathrm{< 0.011\;(0.031)}$  \\
    \hline \hline
    \multirow{2}{*}{TrkId\_VL} & $|\Delta\eta|\mathrm{< 0.01\; (0.01)}$ \\
                               & $|\Delta\phi|\mathrm{< 0.15\;(0.10)}$  \\
    \hline
    \multirow{2}{*}{TrkId\_T} & $|\Delta\eta|\mathrm{< 0.008\; (0.008)}$ \\
                              & $|\Delta\phi|\mathrm{< 0.07\;(0.05)}$ \\
    \hline \hline
    \multirow{2}{*}{CaloIso\_VL} & $\mathrm{ECalIso/E_T <0.2\;(0.2)}$ \\
                                 & $\mathrm{HCalIso/E_T <0.2\;(0.2)}$ \\
    \hline
    \multirow{2}{*}{CaloIso\_T} & $\mathrm{ECalIso/E_T <0.15\;(0.075)}$ \\
                                 & $\mathrm{HCalIso/E_T <0.15\;(0.075)}$ \\
    \hline
    \multirow{2}{*}{CaloIso\_VT} & $\mathrm{ECalIso/E_T <0.05\;(0.05)}$ \\
                                 & $\mathrm{HCalIso/E_T <0.05\;(0.05)}$ \\
    \hline \hline
    TrkIso\_VL                   & $\mathrm{TrkIso/E_T <0.2\;(0.2)}$ \\
    \hline
    TrkIso\_T                   & $\mathrm{TrkIso/E_T <0.15\;(0.075)}$ \\
    \hline
    TrkIso\_VT                   & $\mathrm{TrkIso/E_T <0.05\;(0.05)}$ \\
    \hline \hline
    \multirow{8}{*}{WP80} 		& $\mathrm{H/E < 0.10 (0.05) }$ \\
                               	& $\sigma_{\eta\eta}\mathrm{< 0.01\;(0.03)}$ \\
    							& $|\Delta\eta|\mathrm{< 0.007\; (0.007)}$ \\
                               	& $|\Delta\phi|\mathrm{< 0.06\;(0.03)}$  \\
                               	& $|\frac{1}{E}-\frac{1}{p}|\mathrm{< 0.05\;(0.05)}$  \\
    							& $\mathrm{ECalIso/E_T <0.15\;(0.10)}$ \\
                                & $\mathrm{HCalIso/E_T <0.10\;(0.10)}$ \\
                       			& $\mathrm{TrkIso/E_T <0.05\;(0.05)}$\\
    \hline
 \end{tabular}
 \caption{Summary of requirements applied to electrons in the triggers used for this analysis.
The selection requirements are given for electrons in the barrel (endcap).
The abbrevation in the names means L=Loose, VL=Very Loose, T=Tight, and VT=Very Tight.}
 \label{tab:trg_requirement_def}
\end{table}

\subsection{Utility Triggers}

The lepton selection efficiency measurements are performed using \tnp{} method
on \dyll{} events. In order to use \tnp{} method, we should select pure sample 
of \dyll{} events to reduce systematics due to selecting non-prompt leptons from 
other background processes or Pile-Up. Apart from analysis triggers, for this 
kind of study we do not have to select all events, but pure samples with 
adequate statistics. The single lepton triggers used to collect signal events, 
listed in Table~\ref{tab:trg_singlelepton}, can be used to also select 
\dyll{} events. The leading lepton is likely to be triggered, letting the 
trailing leptons be unbiased sample of leptons that covers wide range of 
kinematic region that streches to the low \pt region.   

In order to estimate backgrounds such as \wjets{} that have a non-prompt 
lepton that passes the full lepton selection, we use "fake rate" method.  
The details of this method are discussed in~\ref{sec:wjets}. 
In this method we define a looser selection and calculate the ratio of the 
events that pass the full selection to the events that pass the looser 
selection using single-lepton events dominated by QCD processes. 
The ratio is called fake rate. The fake rate differs by event kinematics
such as \pt{} and \Eta{} of the lepton, and the \pt{} of the leading jet 
that recoils off the lepton. Given that the leptons in the data sample
collected by the analysis triggers are trigger objects, the tighest possible 
loose definition is the trigger requirements of the analysis triggers.
We devised a set of single-lepton triggers that have the looser or same requirements 
on leptons as the double-lepton triggers. Note that the single lepton triggers 
have tighter requirements on leptons. In order to cover large lepton \pt{} range, 
we use several triggers with different lepton \pt{} thresholds.
Because the jet \pt{} distribution is exponentially falling, 
we need triggers that require corrected leading jet \pt{} be greater than 30 \GeV. 
These triggers give sufficient samples for systematic study on fake rate method. 

%For the data-driven estimation for \dyll{} backgrounds we have alternate method 
%that uses photon + jets events. These events are triggered by photon triggers 
%and the photon triggers used for this analysis are listed in the Table~\ref{tab:}.


\begin{table}[!ht]
  \begin{center}
 {\small
  \begin{tabular} {l|l}
\hline
 Trigger name & L1 seed \\
\hline\hline
 HLT\_Ele8\_CaloIdT\_TrkIdVL\_v[2-5]	& L1\_SingleEG5 		\\ %190456-190738 %190762-191419 %191512-194533
 HLT\_Ele8\_CaloIdT\_CaloIsoVL\_		& L1\_SingleEG7 		\\ 
 TrkIdVL\_TrkIsoVL\_v[12-15]			&  						\\ %190456-190738 %190762-191419 %191512-194533
 HLT\_Ele17\_CaloIdT\_CaloIsoVL\_  		& L1\_SingleEG12		\\ 
 TrkIdVL\_TrkIsoVL\_v[3-6]  			& 						\\ %190456-190738 %190762-191419 %191512-194533
 HLT\_Ele8\_CaloIdT\_CaloIsoVL\_      	& L1\_SingleEG7         \\
 TrkIdVL\_TrkIsoVL\_Jet30\_v[3-7]      	& 						\\ %190456-190738 %190762-191419 %191512-194533
 HLT\_Ele17\_CaloIdT\_CaloIsoVL\_		& L1\_SingleEG12		\\ 
 TrkIdVL\_TrkIsoVL\_Jet30\_v[3-7]		& 						\\ %190456-190738 %190762-191419 %191512-194533
	\hline \hline
 HLT\_Mu8\_v[16-18] 	&  L1\_SingleMu3  		\\ %190456 - 194533
 HLT\_Mu17\_v[3-5]      &  L1\_SingleMu12   	\\ %190456 - 194533   
%	\hline \hline
% HLT\_Photon22\_R9Id90\_HE10\_Iso40\_EBOnly\_v[2-4]					& L1\_SingleEG22		\\ %190456-190738 190762-191419 %191512-194731
% HLT\_Photon36\_R9Id90\_HE10\_Iso40\_EBOnly\_v[2-4]					& L1\_SingleEG22		\\ %190456-190738 190762-191419 %191512-194731
% HLT\_Photon50\_R9Id90\_HE10\_Iso40\_EBOnly\_v[2-4]					& L1\_SingleEG22		\\ %190456-190738 190762-191419 %191512-194731
% HLT\_Photon75\_R9Id90\_HE10\_Iso40\_EBOnly\_v[2-4]					& L1\_SingleEG22		\\ %190456-190738 190762-191419 %191512-194731
% HLT\_Photon90\_R9Id90\_HE10\_Iso40\_EBOnly\_v[2-4]					& L1\_SingleEG22		\\ %190456-190738 190762-191419 %191512-194731
    \hline 
  \end{tabular}
}
  \caption{Utility triggers for fake rate method and zeta method. 
  The identification and isolation requirements for electrons are described in Table~\ref{tab:trg_requirement_def}.
%The identification and isolation requirements for photons are described in Table~\ref{tab:PhotonPlusLeptonTriggerCuts}.
}
%and ``$\zeta$ method " are used for Drell-Yan background estimation.}
   \label{tab:triggers_util}
  \end{center}
\end{table}

%
%\begin{table}[htb]
%  \centering
%  \begin{tabular}{l|c}
%    \hline
%    name                        &  criterion \\
%    \hline \hline 
%    \multirow{1}{*}{R9Id90} 	& $\mathrm{R9 > 0.9 }$ \\
%    \hline 
%    \multirow{1}{*}{HE10} 		& $\mathrm{H/E < 0.1 }$ \\
%    \hline 
%    \multirow{3}{*}{Iso40}     	& $\mathrm{ECalIso} < 4.0 $ \\
%                                & $\mathrm{HCalIso} < 4.0 $ \\
%                                & $\mathrm{TrkIso}  < 4.0 $ \\
%    \hline 
%  \end{tabular}
%   \caption{Summary of requirements applied in the photon triggers used for this analysis.}
%   \label{tab:trg_requirement_def_photon}
%\end{table}

%%%%%%%%%%%%%%%%%%%%%%%%%%%%%%%%%%%%%%%%%%%%%%%%%%%%%%%%%%%%%%%%%%
\section{ Event Primary Vertex }
\begin{itemize}
\item \textcolor{red}{Track reconstruction : Kalman Filter, ...}
\item \textcolor{red}{Vertex reconstruction : vertex finding, vertex fitting, ...}
\item \textcolor{red}{Primary Vertex selection}
\end{itemize}

The primary vertices are reconstructed by Deterministic Annealing clustering of 
tracks \cite{davertex}.

The offline primary vertices are required to be within 24 cm from the center of the 
detector\textcolor{red}{(or beamspot?)} in z direction. 
It should be within 2 cm from the beamspot in the radial direction. 
The degrees of freedom of the vertex fit should be 4 or larger. 
\textcolor{red}{what does this mean exactly?} 

At high luminosity collisions, there are multiple proton-proton interactions 
happening at the same bunch crossing. In those interactions there is usually 
only one interaction that is of interest for our analysis, that triggered that event. 
These interactions tend to be associated with energetic objects, 
while the other interactions are mostly inelastic scatterings that produce soft objects.
Therefore, we choose the event primary vertex by selecting the primary vertex 
with largest scalar sum of $\pt^2$ of tracks associated the vertex. 
The vertices of leptons are required to be very close to the event primary vertex.  



%%%%%%%%%%%%%%%%%%%%%%%%%%%%%%%%%%%%%%%%%%%%%%%%%%%%%%%%%%%%%%%%%%
\section{ Electron }
\begin{itemize}
\item \textcolor{red}{Electron reconstruction : GSF track, seeding, ... }
\item \textcolor{red}{ID  : MVA (list of input variables, trainig samples, cut values)}
\item \textcolor{red}{ISO : list of variables to calculate the final cut variable, cut values  }
\end{itemize}

Electrons are reconstructed using tracks and energy deposit ECAL. 
Since they brem in the tracker, this loss of energy should be taken 
into account in the reconstruction. 

An electron candidate is reconstructed if there is a track and a SC energy deposit
compatible with the track momentum. When there is a random combination of 
a track from $\pi^\pm$ and a SC energy deposit from $\pi^0$ that eventually 
decays to two photons, a fake electron candidate can be made. 
In order to suppress them, we apply electron selection which is composed of 
requirements on identification(track-to-SC matching, 
track fit quality, shower shape, energy loss due to bremstraulung, ratio of hadronic energy
to electromagnetic energy), isolation, impact parameter.  
Other source of fake electrons is a photon conversion to a pair of electron and a positron 
in the material. If the conversion is asymmetric, \textit{i.e.} one particle carries 
most of the photon momentum, that particle can be selected as an electron 
candidate. Thus, we apply additional requirement on the convertion rejection.

% indentification 
For the electron identification we use BDT-based multivariate approach \cite{electronBDT}.  
The trainig is done with 2011 data; \dyll{} events for signal and QCD-dominated events 
collected by the fakerate triggers listed in section~\ref{sec:trigger}. 
In order to increase the separation by selecting and mitigate possible bias 
due to triggers, a set of preselection cuts that are as tight as trigger requirements 
is applied.  

% isolation 

% conversion
In order to suppress the electrons from a conversion from a photon, we reject the electron 
if there is a recontructed conversion vertex where one of the two tracks match with 
the electron, if the probability of the conversion vertex fit is greater than $10^{-6}$, 
\verb|What about cms2.convs_dl()[iconv]> dlMin ?|  
and there are any missing hits in the electron track before the conversion vertex.   

% impact parameters

%%%%%%%%%%%%%%%%%%%%%%%%%%%%%%%%%%%%%%%%%%%%%%%%%%%%%%%%%%%%%%%%%%
\section{ Muon }
\begin{itemize}
\item \textcolor{red}{Muon reconstruction : ...}
\item \textcolor{red}{ID  : list variables and cut values }
\item \textcolor{red}{ISO : MVA (list of input variables, trainig samples, cut values) }
\end{itemize}

%%%%%%%%%%%%%%%%%%%%%%%%%%%%%%%%%%%%%%%%%%%%%%%%%%%%%%%%%%%%%%%%%%
\section{ Jet }
\begin{itemize}
\item \textcolor{red}{Jet reconstruction : anti-kT (dR = 0.5) }
\item \textcolor{red}{Jet energy correction : L1Fastjet/L2/L3 (+ Residual correction in data) }
\item \textcolor{red}{lepton veto ($dR>0.3$) }
\item \textcolor{red}{MVA jet ID(list of input variables, training samples, cut values)}
\item \textcolor{red}{two definitions : (1) jet bin counting($\pt>30~\GeV$) (2) top veto($10<\pt<30~\GeV$) }
% have a plot like this for MVA Jet ID? not sure how I drew this though
% maybe a script is somewhere on uaf
%http://uaf-2.t2.ucsd.edu/~jaehyeok/HWW/PhilJetIDMVA/dev/PhilJetIDMVAefssssssss.pdf
\end{itemize}

%%%%%%%%%%%%%%%%%%%%%%%%%%%%%%%%%%%%%%%%%%%%%%%%%%%%%%%%%%%%%%%%%%
\section{ Missing Transverse Energy }
\begin{itemize}
\item \textcolor{red}{How MET(pfMET and trackMET) is calculated } 
\item \textcolor{red}{mention MET $\phi$ modulation correction }
\item \textcolor{red}{define projected MET (compare signal and bkgd, (ex) \ztt)}
\item \textcolor{red}{cut values ($\textrm{minMET} > 20~\GeV$) }
\end{itemize}

%%%%%%%%%%%%%%%%%%%%%%%%%%%%%%%%%%%%%%%%%%%%%%%%%%%%%%%%%%%%%%%%%%
\section{ Top-tagging }
\begin{itemize}
\item \textcolor{red}{How B-tagging algorithm works and working point
      (TCHEM : Track Counting High Efficiency Medium)}
\item \textcolor{red}{how the discriminating variable is calculated }
\item \textcolor{red}{quote some performance plots }
\item \textcolor{red}{soft-muon tagging requirement }
\end{itemize}
