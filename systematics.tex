
The expected signal and background yields, 
and the shape of the templates can be affected by a number of sources. 
The sources can be categorized as follows. 
\begin{itemize} 
\item Theoretical systematic uncertainties  
\item Instrumental systematic uncertainties  
\item Background estimation systematic uncertainties  
\end{itemize}  
This chapter discusses the list of systematic sources in each catetory, 
how they are estimated, and the result of the estimations. 


%%%%%%%%%%%%%%%%%%%%%%%%%%%%%%%%%%%
\section{Treatment of systematic uncertainties}
 
%%%
\subsubsection{Choice of probability density function(\textit{pdf})}

The systematic uncertainties on the signal and the background yields are 
treated by nuisance parameters. Each nuisance parameter is assumed to follow 
a Probability Density Function(\textit{pdf}).
For most of the nuisance parameters we use the log normal function as a 
\textit{pdf}~\cite{combination_stat}. The functional form of the log normal is 
\begin{eqnarray} 
\rho(\theta) 
&=& 
\frac{1}{\sqrt{2\pi}\ln\kappa}  
\textrm{exp} \left( - \frac{\left( \ln(\theta/\tilde{\theta})\right)^2}
                           {2(\ln\kappa)^2}  \right) 
\frac{1}{\theta}  
\end{eqnarray} 
where $\theta$ is a nuisance parameter, $\tilde{\theta}$ is the best measure 
(mean or median) of the nuisance parameter, and $\kappa$ is the characteristic 
parameter that determines the width of the distribution. 
For a small $\ln\kappa$($\kappa \approx 1$), we can approximate $\ln\kappa \approx \kappa - 1$.
In this case the numerator in the exponent can be effectively treated small as well,
\textit{i.e.}, large values will be suppress by the characteristics of exponential functions,  
and can be approximated in the same way, $\ln (\theta/\tilde{\theta}) \approx \theta/\tilde{\theta} - 1$.
In this approximation, the \textit{pdf} becomes proportional to 
\begin{eqnarray} 
\rho(\theta) 
&\sim&
\textrm{exp} \left( - \frac{\left( \theta/\tilde{\theta} - 1 \right)^2}
                           {2( \kappa - 1)^2}  \right)  
= 
\textrm{exp} \left( - \frac{\left( \theta - \tilde{\theta} \right)^2}
                           {2\tilde{\theta}^2 ( \kappa - 1)^2}  \right).  
\end{eqnarray} 
This equation shows that the exponential function can be 
approximated by a Gaussian in case of $\kappa \approx 1$, and the width 
of the nuisance parameter $\theta$ can be parametrized by $\tilde{\theta}( \kappa - 1)$.
Therefore, $\kappa - 1$ is the relative width with respect to the best 
estimate of the nuisance parameter. In this analysis, we express nuisance parameters 
in term of $\kappa$.

One feature of the log normal function is that the function dies at 0
and we can avoid negative values. This is a big advantage that we can avoid 
the problems such as truncation of \textit{pdf} at 0 as it happens with Gaussian \textit{pdf}.    

There are two nuisance parameters we do not use log-normal as a pdf, 
the signal strength and the normalization of \qqww\ in the shape-based method. 
Both nuisance parameters use a flat(=constant) function as a \textit{pdf}.  
The rationale behind this is that there is no a priori knowledge on those 
nuisances. The nuisance parameter for \qqww\ normalization is chosen 
such that fit can determine the best value of the nuisance using 
the signal-free region dominated by \qqww\ events, without any preference 
of a priori knowledge. 

%%%
\subsubsection{Shape uncertainties}

In the shape-based method, there are systematic sources that can change shapes 
by bin-to-bin migrations in the 2-dimensional templates. 
The normalization uncertainty described by the log-normal or the flat \textit{pdf}  
does not account for this because it changes the overall normalization 
keeping the shape of the distribution unchanged. So, for the sources that can cause 
bin-by-bin migrations, we use alternate shapes.  
The alternate shapes are constructed by changing the source of uncertainty 
by $\pm 1\sigma$. Then, two alternate shapes, up($+1\sigma$) and down($-1\sigma$) shapes,
are used in the statistical machinery following the vertical morphing 
technique~\cite{2011arXiv1103.0354C}. 
This technique uses one additional parameter which follows Gaussian \textit{pdf},
and morphs the alternate shapes such that when the value of the parameter 
is +1(-1), the corresponding variation is $+1\sigma$($-1\sigma$), 
\textit{i.e.}, up(down) shape. 

When the alternate shapes are constructed, 
the correlation between \mT\ and \mll\ is also taken into account naturally.  
Given that there is only one morphing parameter that moves all bins 
by the same amount,  
no matter how the bins are arranged, the correlation is still conserved.  
This is important because we unroll the 2-dimensional template to 1-dimensional 
histograms in order to accommodate the allowed usage of the available statistical tools. 

In the following sections, if there is a related shape uncertainty
caused by a given source, the following plots will be shown in the
in the region where signal is populated($60<\mT<120~\GeV$ and $12<\mll<100~\GeV$); 
the 2-dimensional up/down shapes relative to the central shape 
and \mT\ and \mll\ projections of the up/down/central shapes. 

%%%%%%%%%%%%%%%%%%%%%%%%%%%%%%%%%%%
\section{Theoretical systematic uncertainties}

When data-driven estimation is not applicable, we should rely on the theoretical 
calculation, and it depends on various sources : The PDF and the $\alpha_s$,  
missing higher order corrections, parton shower and underlying events 
and jet bin fraction. 
This section discusses these systematic uncertainties. 

\subsection{PDF+$\alpha_s$}

The parton distribution function(PDF) uncertainty together with $\alpha_s$ uncertainty 
that affects lepton acceptance and the efficiency of all cuts 
is estimated following the prescription recommended by PDF4LHC group~\cite{Botje:2011sn}. 
We take the three PDF sets, MSTW2008, CT10 and NNPDF, 
and propagate the uncertainty of each prescription. 
The envelop of the uncertainties is taken as total uncertainty. 

The PDF+$\alpha_s$ uncertainty is divided into two groups depending on the 
production source($q\bar{q}$ or gg). 
The processes in the same group are assumed to be 100 \% correlated 
while the processes in the different groups are assumed to be 100 \% uncorrelated.
The uncertainty for \ggH\ ranges from 7~\% to 12~\%,  
and the uncertainty for other signal processes is 5~\%.
All other processes, \ggww, \qqww, \vv, \wgamma\ and \wgammastar\  has 4~\% of uncertainty.

In the shape-based method, the PDF+$\alpha_s$ shape uncertainty 
is considered for \qqww\ and \ggww.  
Following the method described in~\cite{pdfAN}, we take an envelop of different PDF 
sets(NNPDF2.0, CT10(CTEQ) and MSTW2008), and measure variation with respect to the default 
PDF set(LO CTEQ6L1 PDF) used in MC generation. 
Fig.~\ref{fig:alter_pdfqqww} and~\ref{fig:alter_pdfggww} show the up/down 
shapes in 2-dimensional template and its projection to \mT\ and \mll\ axes
in the signal region. 
The variation is almost flat on the 2D plane with a size less than 5~\%. 
The impact of changing from normalization uncertainty to shape systematics on PDF 
is less than 1~\% to the expected significance.
%
\begin{figure}[htp] 
\centering 
\begin{tabular}{c} 
\includegraphics[width=0.8\textwidth]{figures/histo_qqWW_CMS_hww_PDFqqWW_0j_zoom.pdf} 
\end{tabular} 
\caption{ \qqww\ PDF+$\alpha_s$ in the 0-jet category}
\label{fig:alter_pdfqqww} 
\end{figure} 
%
\begin{figure}[htp] 
\centering 
\begin{tabular}{c} 
\includegraphics[width=0.8\textwidth]{figures/histo_ggWW_CMS_hww_PDFggWW_0j_zoom.pdf} 
\end{tabular} 
\caption{ \ggww\ PDF+$\alpha_s$ in the 0-jet category.}
\label{fig:alter_pdfggww} 
\end{figure} 

%\subsubsection{QCD Renormalization and factorization scale variation} 
\subsection{Missing higher order corrections} 

The cross section of a particular process is calculated by a perturbation expansion, 
where the first few orders are considered due to complication of calculation 
with higher order terms. In case of SM Higgs production, the cross sections is calculated 
up to NNLO in QCD as shown in Table~\ref{tab:Higgs_XS_8TeV_order}. Missing higher orders 
in the calculation should be accounted for in some way, and this is done by 
varying the renormalization scale(\mR) and the factorization scale(\mF) 
by a factor of 2 or 1/2~\cite{Dittmaier:1318996}. 

For the \ggH\ and \qqww\ processes for $\mHi>200~\GeV$, 
this effect is expressed in terms of uncertainty to the 
exclusive jet bins and will be discussed in detail in section~\ref{sec:jetbinfrac}. 
For other process, the uncertainty due to QCD scale variation ranges from 1 - 4~\%.   

In the shape-based method, the effect of the \mR\ and \mF\ variations to shapes 
is considered for \qqww. The MC@NLO 4.0~\cite{Frixione:2002ik} is used for the matrix element calculation 
with different choice of scales. The up variation corresponds to $\mR=0.5\mu$ and $\mR=2.0\mu$,
and the down variation corresponds to $\mR=2.0\mu$ and $\mR=0.5\mu$ where $\mu$ is the nominal 
scale value. Fig.~\ref{fig:alter_qqwwnlo} shows the up and down  
shapes in the 2-dimensional template and its projection to \mT\ and \mll\ axes
in the signal region.

Another way of include the effect of missing higher order corrections 
is to compare results calculated in different orders. For \qqww\ and \topbkg\ 
backgrounds we have the shape systematics that compare two different simulations. 
For \qqww\ an alternate sample generated by MC@NLO 4.0 which is calculated up to NLO
is used to define a up variation, 
and the mirror with respect to the central shape(Madgraph 5.1~\cite{Madgraph5p1}) which is calculated up to LO 
is used as a down variation.
Fig.~\ref{fig:alter_qqww} shows the up/down alternate shapes. 
The MC@NLO sample uses different parton shower model, Herwig++~\cite{Corcella:2000bw}, 
as opposed to 
Pythia 6.4~\cite{Sjostrand:2006za} used by the default sample, 
so it also accounts for the effect of uncertainty 
to the modeling of parton shower. 
For \topbkg\ an alternate sample generated by Madgraph 5.1 which is calculated up to LO
is used to define a up variation,
and the mirror with respect to the central shape(Powheg~\cite{Frixione:2007vw}) which is calculated up to NLO 
is used as a down variation.
Fig.~\ref{fig:alter_qqww} shows the up/down alternate shapes. 

%
\begin{figure}[htp]
\centering
\begin{tabular}{c}
\includegraphics[width=0.8\textwidth]{figures/histo_qqWW_CMS_hww_MVAWWNLOBounding_0j_zoom.pdf}
\end{tabular}
\caption{ \qqww\ QCD scale variation in the 0-jet category.}
\label{fig:alter_qqwwnlo}
\end{figure}
%
\begin{figure}[htp]
\centering
\begin{tabular}{c}
\includegraphics[width=0.8\textwidth]{figures/histo_qqWW_CMS_hww_MVAWWBounding_0j_zoom.pdf}
\end{tabular}
\caption{ \qqww\ Madgraph vs. MC@NLO in the 0-jet category.}
\label{fig:alter_qqww}
\end{figure}
%
\begin{figure}[htp]
\centering
\begin{tabular}{c}
\includegraphics[width=0.8\textwidth]{figures/histo_Top_CMS_hww_MVATopBounding_1j_zoom.pdf}
\end{tabular}
\caption{ \topbkg\ Powheg vs. Madgraph in 1-jet category.}
\label{fig:alter_top}
\end{figure}

%%%
\subsection{Parton shower and underlying events}

% how UE is estimated?
The systematic uncertainty on modeling of parton shower(PS) and 
underlying events(UE)~\cite{Chatrchyan:2011id,Chatrchyan:2012tb}
is evaluated by comparing different generators using different PS and UE models.
A simulation chain that uses Powheg for ME calculation interfaced with Pythia 6.4 for PS 
is compared with a simulation chain that uses MC@NLO 4.0 for ME calculation 
interfaced with Herwig++ for PS. 
%\textcolor{red}{what about the UE models? How are they different?}
In order to exclude the effect of using different ME calculators, 
the Higgs \pt\ is normalized to the reference distribution~\cite{Dittmaier:1318996}.
The Table~\ref{tab:UEPS} shows the $\kappa$ values of the PS/UE uncertainty 
for a given \mHi. 
%
\begin{table}[htp] 
\begin{center} 
\begin{tabular}{c|ccc} 
\hline 
\mHi\ [\GeV]  & 0-jet & 1-jet & 2-jet \\
\hline \hline 
115 & 0.941 & 1.128 & 1.212 \\   
120 & 0.940 & 1.110 & 1.293 \\   
130 & 0.937 & 1.113 & 1.237 \\   
140 & 0.941 & 1.104 & 1.168 \\   
150 & 0.942 & 1.093 & 1.156 \\   
160 & 0.943 & 1.084 & 1.138 \\   
170 & 0.946 & 1.075 & 1.108 \\   
180 & 0.947 & 1.067 & 1.092 \\   
190 & 0.948 & 1.068 & 1.083 \\   
200 & 0.952 & 1.055 & 1.059 \\   
250 & 0.955 & 1.058 & 0.990 \\   
300 & 0.958 & 1.061 & 0.942 \\   
350 & 0.964 & 1.068 & 0.889 \\   
400 & 0.966 & 1.078 & 0.856 \\   
450 & 0.954 & 1.092 & 0.864 \\   
500 & 0.946 & 1.102 & 0.868 \\   
550 & 0.931 & 1.117 & 0.861 \\   
600 & 0.920 & 1.121 & 0.872 \\   
\hline 
\end{tabular} 
\caption{$\kappa$ values of systematic uncertainty due to modeling of parton showering and underlying events.} 
\label{tab:UEPS} 
\end{center} 
\end{table} 


%%%
\subsection{Jet Bin Fractions}
\label{sec:jetbinfrac}

The analysis is optimized in the different jet categories, 
and the jets are counted with the requirement that  
its transverse momentum is greater than 30~\GeV. For the processes that the yields in the 
different jet categories are estimated by data-driven methods, there is not related uncertainty 
because the fraction of events in the different jet categories comes from data. 
On the other hand, for the processes taken from simulation we need to account for this effect.    
The fraction of events that falls into a particular jet category is 
affected by the kinematics of events, and the kinematics of events is affected 
by missing higher order terms. 
In order to account for this effect, we evaluate the relevant uncertainty by varying 
QCD scales without any additional selection cuts.

We first get the fraction of events in the difference jet categories, 
$f_\textrm{0-jet}$ for $\textrm{0-jet}$, $f_\textrm{1-jet}$ for $\textrm{1-jet}$ 
and $f_\textrm{2-jet}$ for $\ge\textrm{2-jet}$ where the fractions are defined as 
\begin{eqnarray} 
f_\textrm{0-jet} 
&=&  
\frac{\sigma_{\ge\textrm{0-jet}} - \sigma_{\ge\textrm{1-jet}}}{\sigma_{\ge\textrm{0-jet}}} \\
f_\textrm{1-jet} 
&=&  
\frac{\sigma_{\ge\textrm{1-jet}} - \sigma_{\ge\textrm{2-jet}}}{\sigma_{\ge\textrm{0-jet}}} \\
f_\textrm{2-jet} 
&=&  
\frac{\sigma_{\ge\textrm{2-jet}}}{\sigma_{\ge\textrm{0-jet}}}.
\end{eqnarray}  
where $\sigma_{\ge\textrm{0/1/2-jet}}$ is the inclusive cross section 
with number of jets $\ge 0/1/2$.

The systematic uncertainties on the inclusive cross sections, 
$\sigma_{\ge\textrm{0-jet}}$, $\sigma_{\ge\textrm{1-jet}}$ and $\sigma_{\ge\textrm{2-jet}}$ are 
manifested by $\kappa_{\ge\textrm{0-jet}}$, $\kappa_{\ge\textrm{1-jet}}$ and $\kappa_{\ge\textrm{2-jet}}$, respectively.
The central values of the inclusive cross sections are calculated 
using MSTW2008 NLO PDF at the QCD scales, $\mR=\mF=\mHi/2$.
The uncertainty on the $\sigma_{\ge\textrm{0-jet}}$, $\kappa_{\ge\textrm{0-jet}}$, is taken from the 
CERN Yellow Report~\cite{Dittmaier:1318996}, and the uncertainties 
on $\sigma_{\ge\textrm{1-jet}}$ and $\sigma_{\ge\textrm{2-jet}}$, 
$\kappa_{\ge\textrm{1-jet}}$ and $\kappa_{\ge\textrm{2-jet}}$, are 
calculated by the following QCD scale variations using MCFM~\cite{Campbell:2010ff} : 
\begin{itemize}
\item $\mF = \mHi $, $\mR = \mHi$
\item $\mF = \mHi / 4$, $\mR = \mHi / 4$
\item $\mF = \mHi $, $\mR = \mHi / 2$
\item $\mF = \mHi / 2$, $\mR= \mHi$.
\end{itemize}
following the recommendation by Higgs Cross Section WG~\cite{Dittmaier:2012vm}. 
The largest positive($\Delta_{+}$) and negative($\Delta_{-}$) uncertainties 
to $\sigma_{\ge\textrm{1-jet}}$ and $\sigma_{\ge\textrm{2-jet}}$ are taken,
and they are symmetrized to be expressed in terms of $\kappa$. 
The symmetrization is done by 
\begin{eqnarray} 
  \kappa_{\mathrm{symmetrized}} = \sqrt{e^{\Delta_{+}} \times e^{\Delta_{-}}} ,
\end{eqnarray} 
where the $\kappa_{\mathrm{symmetrized}}$ is the resultant $\kappa$
for the asymmetric uncertainties, $\Delta_{+}$ and $\Delta_{-}$.
Using the values in Table~\ref{tab:jetbinincl}, we can convert the inclusive jet bin 
uncertainties to the exclusive jet bin uncertainties by the formulae shown in 
Table~\ref{tab:jetbinexcl}~\cite{Dittmaier:2012vm}.
\begin{table}[!htbp]
\begin{center}
\begin{tabular}{c|ccc}

\hline
Nuisance Parameter & $\kappa$'s for 0-jet bin   & $\kappa$'s for 1-jet bin  & $\kappa$'s for 2-jet bin                       \\
\hline \hline
$\theta_{\textrm{from } \ge \textrm{0-jet}}$  & $ (\kappa_{\geq 0})^{\frac{1}{f_\textrm{0-jet}}}$    & $1.0$ & $1.0$                                            \\
$\theta_{\textrm{from } \ge \textrm{1-jet}}$  & $(\kappa_{\geq 1})^{- \frac{f_\textrm{1-jet}+f_\textrm{2-jet}}{f_\textrm{0-jet}}}$ & $(\kappa_{\geq 1})^{\frac{f_\textrm{1-jet}+f_\textrm{2-jet}}{f_\textrm{1-jet}}}$  & $1.0$                                            \\
$\theta_{\textrm{from } \ge \textrm{2-jet}}$  & $1.0$ & $(\kappa_{\geq 2})^{- \frac{f_\textrm{2-jet}}{f_\textrm{1-jet}}} $     & $\kappa_{\geq 2}$ \\
\hline

\end{tabular}
\caption{Table of formulae expressing the $\kappa$ values for the systematic uncertainties on the
jet bin fractions due to the missing higher order corrections on 
$\sigma_{\ge\textrm{0-jet}}$, $\sigma_{\ge\textrm{1-jet}}$ and $\sigma_{\ge\textrm{2-jet}}$,
in terms of the $\kappa$ values for these cross sections, 
$\kappa_{\ge\textrm{0-jet}}$, $\kappa_{\ge\textrm{1-jet}}$ and $\kappa_{\ge\textrm{2-jet}}$, respectively, 
and the jet bin fractions, 
$f_{\textrm{0-jet}}$, $f_{\textrm{1-jet}}$ and $f_{\textrm{2-jet}}$.
}
\label{tab:jetbinformula}
\end{center}
\end{table}

For \ggH\ process, the numerical values of $\kappa_{\ge\textrm{0-jet}}$, $\kappa_{\ge\textrm{1-jet}}$ 
and $\kappa_{\ge\textrm{2-jet}}$ are summarized in Table~\ref{tab:jetbinincl}.
%
\begin{table}[!htbp]
\begin{center}
\begin{tabular}{c|ccc}
\hline
\mHi[\GeV]    &     $\kappa_{\ge\textrm{0-jet}}$        &   $\kappa_{\ge\textrm{1-jet}}$        &     $\kappa_{\ge\textrm{2-jet}}$       \\
\hline\hline
 115 & $ 1.106$  & $ 1.226$  & $ 1.149$  \\
 120 & $ 1.104$  & $ 1.224$  & $ 1.120$  \\
 130 & $ 1.100$  & $ 1.230$  & $ 1.117$  \\
 140 & $ 1.096$  & $ 1.220$  & $ 1.129$  \\
 150 & $ 1.095$  & $ 1.220$  & $ 1.124$  \\
 160 & $ 1.095$  & $ 1.221$  & $ 1.199$  \\
 170 & $ 1.090$  & $ 1.222$  & $ 1.175$  \\
 180 & $ 1.089$  & $ 1.218$  & $ 1.171$  \\
 190 & $ 1.087$  & $ 1.217$  & $ 1.171$  \\
 200 & $ 1.087$  & $ 1.213$  & $ 1.197$  \\
 250 & $ 1.083$  & $ 1.208$  & $ 1.230$  \\
 300 & $ 1.082$  & $ 1.208$  & $ 1.205$  \\
 350 & $ 1.090$  & $ 1.207$  & $ 1.209$  \\
 400 & $ 1.075$  & $ 1.195$  & $ 1.195$  \\
 450 & $ 1.078$  & $ 1.194$  & $ 1.196$  \\
 500 & $ 1.087$  & $ 1.188$  & $ 1.174$  \\
 550 & $ 1.089$  & $ 1.191$  & $ 1.194$  \\
 600 & $ 1.090$  & $ 1.187$  & $ 1.192$  \\

\hline
\end{tabular}
\caption{ $\kappa$ values for the systematic uncertainties due to missing higher order corrections
for the inclusive \ggH\ production cross section, 
$\sigma_{\ge\textrm{0-jet}}$, $\sigma_{\ge\textrm{1-jet}}$ and $\sigma_{\ge\textrm{2-jet}}$. 
The corresponding $\kappa$s are $\kappa_{\ge\textrm{0-jet}}$, $\kappa_{\ge\textrm{1-jet}}$ 
and $\kappa_{\ge\textrm{2-jet}}$, respectively. 
}
\label{tab:jetbinincl}
\end{center}
\end{table}
Combining information in Table~\ref{tab:jetbinincl} and Table~\ref{tab:jetbinformula} 
we obtain the final $\kappa$'s for \ggH\ process as shown in Table~\ref{tab:jetbinexcl}.
%
\begin{table}[!htbp]
\begin{center}
\begin{tabular}{c|cc|cc|c}

\hline
\multirow{2}{*}{\mHi [\GeV]} 
               & \multicolumn{2}{c|}{0-jet bin}  & \multicolumn{2}{c|}{1-jet bin}  & 2-jet bin \\ 
               & $\kappa_{\mathrm{from } \ge \textrm{0-jet}}$ & $\kappa_{\mathrm{from } \ge \textrm{1-jet}}$ 
               & $\kappa_{\mathrm{from } \ge \textrm{1-jet}}$ & $\kappa_{\mathrm{from } \ge \textrm{2-jet}}$  
               & $\kappa_{\mathrm{from } \ge \textrm{2-jet}}$   \\
\hline \hline
 115 & $ 1.16$  & $ 0.92$  & $ 1.28$  & $ 0.97$  & $ 1.15$  \\
 120 & $ 1.16$  & $ 0.92$  & $ 1.28$  & $ 0.97$  & $ 1.12$  \\
 130 & $ 1.15$  & $ 0.91$  & $ 1.29$  & $ 0.98$  & $ 1.12$  \\
 140 & $ 1.15$  & $ 0.91$  & $ 1.28$  & $ 0.97$  & $ 1.13$  \\
 150 & $ 1.15$  & $ 0.90$  & $ 1.28$  & $ 0.97$  & $ 1.12$  \\
 160 & $ 1.15$  & $ 0.90$  & $ 1.28$  & $ 0.96$  & $ 1.20$  \\
 170 & $ 1.15$  & $ 0.89$  & $ 1.28$  & $ 0.96$  & $ 1.18$  \\
 180 & $ 1.15$  & $ 0.89$  & $ 1.28$  & $ 0.96$  & $ 1.17$  \\
 190 & $ 1.15$  & $ 0.88$  & $ 1.28$  & $ 0.96$  & $ 1.17$  \\
 200 & $ 1.15$  & $ 0.88$  & $ 1.27$  & $ 0.96$  & $ 1.20$  \\
 250 & $ 1.16$  & $ 0.86$  & $ 1.27$  & $ 0.96$  & $ 1.17$  \\
 300 & $ 1.17$  & $ 0.84$  & $ 1.27$  & $ 0.95$  & $ 1.20$  \\
 350 & $ 1.20$  & $ 0.83$  & $ 1.27$  & $ 0.95$  & $ 1.21$  \\
 400 & $ 1.17$  & $ 0.82$  & $ 1.26$  & $ 0.95$  & $ 1.20$  \\
 450 & $ 1.19$  & $ 0.81$  & $ 1.26$  & $ 0.95$  & $ 1.20$  \\
 500 & $ 1.22$  & $ 0.80$  & $ 1.25$  & $ 0.95$  & $ 1.17$  \\
 550 & $ 1.24$  & $ 0.78$  & $ 1.26$  & $ 0.95$  & $ 1.19$  \\
 600 & $ 1.25$  & $ 0.78$  & $ 1.26$  & $ 0.94$  & $ 1.19$  \\

\hline

\end{tabular}
\caption{ Table of $\kappa$ values for the systematic uncertainties for the jet bin 
fractions due to missing higher order corrections for the total inclusive Higgs
cross section, the inclusive Higgs+1jet cross section, and the inclusive Higgs+2jet
cross section. }
\label{tab:jetbinexcl}
\end{center}
\end{table}

For \qqww\ process, the inclusive cross sections, $\sigma_{\ge\textrm{0-jet}}$,
$\sigma_{\ge\textrm{1-jet}}$ and $\sigma_{\ge\textrm{2-jet}}$ are evaluated using 
MC@NLO 4.0~\cite{Frixione:2002ik} 
and the corresponding uncertainties are estimated by combination of  
QCD scale variation and the comparison with ALPGEN~\cite{Mangano:2002ea}. 
The uncertainties to the above inclusive cross sections are 3~\%,  6~\% and 42~\% for  
$\sigma_{\ge\textrm{0-jet}}$, $\sigma_{\ge\textrm{1-jet}}$ 
and $\sigma_{\ge\textrm{2-jet}}$, respectively~\cite{Dittmaier:2012vm}.  
The jet bin fractions calculated using the inclusive cross sections are 
$f_{\textrm{0-jet}}=0.70$, $f_{\textrm{1-jet}}=0.22$ and $f_{\textrm{2-jet}}=0.08$.
Inserting these numbers into the formulae in Table~\ref{tab:jetbinformula}, 
we obtain the $\kappa$'s for the jet bin fraction uncertainties for \qqww\ process
as shown in Table~\ref{tab:jetbinexcl_ww}. 
%
\begin{table}[!htbp]
\begin{center}
\begin{tabular}{cc|cc|c}
\hline
\multicolumn{2}{c|}{0-jet bin}  & \multicolumn{2}{c|}{1-jet bin}  & 2-jet bin \\
$\kappa_{\mathrm{from } \ge \textrm{0-jet}}$ & $\kappa_{\mathrm{from } \ge \textrm{1-jet}}$
& $\kappa_{\mathrm{from } \ge \textrm{1-jet}}$ & $\kappa_{\mathrm{from } \ge \textrm{2-jet}}$
& $\kappa_{\mathrm{from } \ge \textrm{2-jet}}$   \\
\hline \hline
$1.042$  & $ 0.978$  & $ 1.076$  & $ 0.914$  & $ 1.42$  \\
\hline
\end{tabular}
\caption{ Table of $\kappa$ values for the systematic uncertainties for the jet bin
fractions due to missing higher order corrections for the total inclusive \qqww\
cross section, the inclusive \qqww+1jet cross section, and the inclusive \qqww+2jet
cross section. }
\label{tab:jetbinexcl_ww}
\end{center}
\end{table}

%\subsection{Theoretial extrapolation uncertainty for WW}


%%%%%%%%%%%%%%%%%%%%%%%%%%%%%%%%%%%
\section{Instrumental Systematic Uncertainties} 
%\begin{itemize} 
%\item Luminosity : for 8 TeV which luminosity uncertainty to quote? %4.4\% of 2.6\%? 
%      How is the uncertainty estimated?
%\item PU : ~1\% so neglected    
%\item Lepton momentum scale and resolution (shape for 2D) 
%      How is the uncertainty estimated?
%\item Lepton efficiency (shape for 2D)    
%      How is the uncertainty estimated?
%\item MET resolution (shape for 2D)    
%      How is the uncertainty estimated?
%\item Jet energy scale (shape for 2D)     
%      How is the uncertainty estimated? for shape $\pm5\%$ for jet bin migration 
%\item  MC statistics shape uncertainty : mention bin-by-bin treatment gives consistent result  
%\end{itemize} 


%%%
\subsubsection{Luminosity}

The luminosity at CMS is measured by the hadronic forward calorimeter(HF)
and the silicon pixel detector. Thanks to small dependence on experimental 
conditions such as pileup, the counting of the pixel clusters is chosen 
for the precision luminosity measurement~\cite{CMS-PAS-LUM-13-001}. 
The measured luminosity is calibrated by Van Der Meer Scans at the 
ISR~\cite{CMS-PAS-LUM-13-001}.
The dominant source of uncertainty is the choice of fit function 
to model the bunch shapes.  

The total estimated uncertainties are 2.6~\%~\cite{CMS-PAS-LUM-13-001} at 8~\TeV\ 
and 2.2~\% at 7~\TeV~\cite{Chatrchyan:2013oda}.  

%%%
\subsubsection{Lepton momentum scale and resolution}

The lepton momentum scale and resolution can affect the selection efficiency 
for the cuts applied on lepton momentum or variables constructed using 
lepton momenta. The size of the uncertainties of both sources is estimated 
by comparing Z invariant mass shape between simulation and data in the \SF\ 
final state. The momentum scale is responsible for the location of the 
invariant mass peak, and the momentum resolution is responsible for the width 
of the distribution. 

The measured uncertainties are 1.5~\%(3~\%) for electrons in barrel(endcap)  
and 1.0~\%(1.7~\%) for muons in barrel(endcap). For the cut-based method 
we use the average over barrel and endcap, 2~\% for electron and 1.5~\% for muon. 
For the shape-based method, new \mT\ and \mll\ are calculated using the 
lepton momenta scaled by the measured resolutions.
The up shape is made by adding the smeared momentum resolution to both leptons,
and the down shape is made by subtracting the smeared momentum resolution to both leptons.
Fig.~\ref{fig:alter_lepres} shows the corresponding up/down alternate shapes 
for \qqww\ in the 0-jet \DF\ category. 
%
\begin{figure}[htp]
\centering
\begin{tabular}{c}
\includegraphics[width=0.8\textwidth]{figures/histo_qqWW_CMS_hww_MVALepResBounding_0j_zoom.pdf}
\end{tabular}
\caption{Alternate shapes for lepton momentum scale and resolution for \qqww\ in the 0-jet \DF\ category.}
\label{fig:alter_lepres}
\end{figure}


%%%
\subsubsection{Lepton efficiency} 

The lepton selection and trigger efficiencies are measured by the Tag-and-Probe method
as described in chapter~\ref{ch:efficiency_measurement}. In this method, uncertainties come from 
the determination of the background contribution and the modeling of the signal 
and background shapes in the likelihood fit. An additional uncertainty which is 
prominent in the low \pt\ region in the electron case 
comes from the possible bias by using the N-1 technique. 
This is described in section~\ref{sec:electroneff} as well as the corresponding 
uncertainties in Table~\ref{tab:eff_electron_nmsyst}.

The estimated uncertainties are 1.5~\% for muons and 2~\% for electrons, 
and we use 3~\% and 4~\% for \mumu\ and \ee\ events, respectively, 
in the cut-based method. 
For the shape-based method, alternate shapes are constructed by
scaling up/down the lepton efficiencies using the systematics sources
as a function of lepton \pt\ and \Eta.
Fig.~\ref{fig:alter_lepeff} shows the corresponding up/down alternate shapes
for \qqww\ in the 0-jet \DF\ category. 

%
\begin{figure}[htp]
\centering
\begin{tabular}{c}
\includegraphics[width=0.8\textwidth]{figures/histo_qqWW_CMS_hww_MVALepEffBounding_0j_zoom.pdf}
\end{tabular}
\caption{Alternate shapes for lepton efficiency for \qqww\ in the 0-jet \DF\ category. }
%\textcolor{red}{looks too small at low \mT\ and \mll.}}
\label{fig:alter_lepeff}
\end{figure}


%%%
\subsubsection{MET resolution} 

The mis-modeling of \met\ by simulation can introduce a systematic uncertainty 
as we select events that pass \met\ and \met-related selections. 
The uncertainty is measured 
by comparing data and simulation using \dyll\ events. To account for the 
difference between data and simulation, an additional Gaussian smearing for 
the individual x and y components of \met\ are needed. 
For PF \met, the size of Gaussian smearing is 3.2 and 3.6~\GeV\ %and 4.3 
for 0-jet and 1-jet categories, respectively.  
For \trkmet, the size of Gaussian smearing is 1.0 and 4.5~\GeV\ %and 6.0 
for 0-jet and 1-jet categories, respectively.  

For the cut-based method, the resultant effect of smearing to yields is around 2~\%. 
For the shape-based method, the up/down alternate shapes are constructed 
with the new \mT\ calculated using the smeared x and y components of \met.
Fig.~\ref{fig:alter_lepeff} shows the corresponding up/down alternate shapes
for \qqww\ in the 0-jet \DF\ category. 

%
\begin{figure}[htp]
\centering
\begin{tabular}{c}
\includegraphics[width=0.8\textwidth]{figures/histo_qqWW_CMS_hww_MVAMETResBounding_0j_zoom.pdf}
\end{tabular}
\caption{Alternate shapes for \met\ resolution for \qqww\ in the 0-jet \DF\ category.}
\label{fig:alter_metres}
\end{figure}

%%%
\subsubsection{Jet energy scale} 

The analysis is optimized in jet categories, and the jet selection is done by 
requiring \pt\ to be greater than 30~\GeV. Thus, uncertainty on the 
energy of jets can lead to migration between different jet categories.  
From the studies on the PF jet energy resolution done in~\cite{Chatrchyan:2011ds}, 
the jet energy resolution ranges from 3~\% to 5~\% as the \Eta\ of jets increase. 
We take 5 \% for whole \Eta\ range as a conservative choice.   

For the cut-based method, the resultant effect of 5~\% of jet energy resolution 
to yields is around 2~\%. 
For the shape-based method, the up/down alternate shapes are constructed
by scaling the jet transverse momentum by $\pm 5$~\%.
Fig.~\ref{fig:alter_jes} shows the corresponding up/down alternate shapes
for \qqww\ in the 0-jet \DF\ category. 

%
\begin{figure}[htp]
\centering
\begin{tabular}{c}
\includegraphics[width=0.8\textwidth]{figures/histo_qqWW_CMS_hww_MVAJESBounding_0j_zoom.pdf}
\end{tabular}
\caption{Alternate shapes for jet energy resolution for \qqww\ in the 0-jet \DF\ category.}
\label{fig:alter_jes}
\end{figure}

%%%
\subsubsection{Statistics of simulated samples} 

The limited statistics of the available simulation samples should be taken 
into account as a source of systematic uncertainty. 
For the cut-based method, overall statistical uncertainty of the sample 
after the final selection is taken into account. 
For the shape-based method, the up/down alternate shapes are constructed 
by adding/subtracting the size of statistical uncertainty in each bin. 
This does not allow bin-by-bin fluctuation in the statistical mechinery 
because all bins are either up or down by the statistical uncertainties 
in each bin. The ideal method should be considering each bin independently,
but this requires an extensive CPU consumption. 
We checked the expected and the observed significances using both approaches,
and found that the results are compatible within a few \%. 
Thus, we use the former approach. 
Fig.~\ref{fig:alter_stat} shows the corresponding up/down alternate shapes
for \ggww\ in the 0-jet category. 
%
\begin{figure}[htp]
\centering
\begin{tabular}{c}
\includegraphics[width=0.8\textwidth]{figures/histo_ggWW_CMS_hww_of_0j_MVAggWWStatBounding_8TeV_0j_zoom.pdf}
\end{tabular}
\caption{Alternate shapes for MC statistics for \ggww\ in the 0-jet \DF\ category }
\label{fig:alter_stat}
\end{figure}



%%%%%%%%%%%%%%%%%%%%%%%%%%%%%%%%%%%
\section{Background Estimation Uncertainty} 

The expected background contributions in the signal region are estimated by 
data-driven methods or taken from simulation after some data corrections.  
All procedures and the source of systematic uncertainties are discussed 
in chapter~\ref{ch:background_estimation}. These uncertainties are related 
to the normalization of each background. But, there is a shape systematic uncertainty 
that can cause a variation of shapes. 

\subsubsection{\Wjets\ alternate shapes} 

As described in section~\ref{sec:wjets}, the \Wjets\ background is estimated 
by a data-driven method which measures the fake rate in the QCD di-jet sample. 
One of the systematic sources is the variation of the away jet \pt,
and the uncertainty of 30~\% is assigned to take this account. 
In the shape-based method, the affect of the varying the away jet \pt\ 
to the shape is considered as well. 
The alternate up shape is constructed using alternate jet \pt\ thresholds, 
15~\GeV for muons and 20~\GeV for electrons, and the relative difference 
in shape is taken. The alternate down shape is taken as a mirror of 
the up shape with respect to the central shape. 
Fig.~\ref{fig:alter_wjets} shows the corresponding up/down alternate shapes
for \Wjets\ when muon is an FO. 
%
\begin{figure}[htp]
\centering
\begin{tabular}{c}
\includegraphics[width=0.8\textwidth]{figures/histo_WjetsM_CMS_hww_MVAWMBounding_0j_zoom.pdf}
\end{tabular}
\caption{Alternate shapes for \WjetsM. }
\label{fig:alter_wjets}
\end{figure}


%%%%%%%%%%%%%%%%%%%%%%%%%%%%%%%%%%% \section{Summary table?} 
